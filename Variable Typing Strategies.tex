% Uncomment for handout
\def\HANDOUT{}


\ifdefined\HANDOUT
\documentclass[handout]{beamer}
\usepackage{pgfpages}
\pgfpagesuselayout{4 on 1}[letterpaper,landscape,border shrink=5mm]
\else
\documentclass{beamer}
\fi

\mode<presentation>
{
  \usetheme{Warsaw}
  \definecolor{sered}{rgb}{0.78, 0.06, 0.18}
  \definecolor{richblack}{rgb}{0.0, 0.0, 0.0}
  \setbeamercolor{structure}{fg=sered,bg=richblack}
  %\setbeamercovered{transparent}
}


\usepackage[english]{babel}
\usepackage[latin1]{inputenc}
\usepackage{times}
\usepackage[T1]{fontenc}
\usepackage{tikz}
\usepackage{graphicx}
\usepackage[export]{adjustbox}
\usepackage{fancyvrb}
\usepackage{amsmath}
\usepackage{amssymb}
\usepackage{esvect}

\makeatletter
\newcommand{\imagesource}[1]{{\centering\hfill\break\hbox{\scriptsize Image Source:\thinspace{\tiny\itshape #1}}\par}}
\newcommand{\image}[3][\@nil]{%
        \def\tmp{#1}%
        \begin{center}
        \ifx\tmp\@nnil
            \includegraphics[max height = 0.55\textheight, max width = \textwidth]{images/#2}
        \else
            \includegraphics[max height = 0.50\textheight, max width = \textwidth]{images/#2}
            \linebreak
            #1
        \fi
        \linebreak
        {\tiny Image Source:\thinspace{\tiny #3}}
        \end{center}
}

\newenvironment{code}{%
 \VerbatimEnvironment
 \begin{adjustbox}{max width=\textwidth, max height=0.7\textheight}
 \begin{BVerbatim}
  }{
  \end{BVerbatim}
 \end{adjustbox}
}


\newenvironment{scaled}{%
 \begin{adjustbox}{max width=\textwidth, max height=0.7\textheight}
 }{
 \end{adjustbox}
}

\title{Variable Typing Strategies}


\author{Robert Lowe}

\institute[Southeast Missouri State University] % (optional, but mostly needed)
{
  Department of Computer Science\\
  Southeast Missouri State University
}

\date[]{}
\subject{}

\pgfdeclareimage[height=1.0cm]{university-logo}{images/semo-logo}
\logo{\pgfuseimage{university-logo}}



\AtBeginSection[]
{
  \begin{frame}<beamer>{Outline}
    \tableofcontents[currentsection]
  \end{frame}
}


\begin{document}

\begin{frame}
  \titlepage
\end{frame}

\begin{frame}{Outline}
  \tableofcontents
\end{frame}


% Structuring a talk is a difficult task and the following structure
% may not be suitable. Here are some rules that apply for this
% solution: 

% - Exactly two or three sections (other than the summary).
% - At *most* three subsections per section.
% - Talk about 30s to 2min per frame. So there should be between about
%   15 and 30 frames, all told.

% - A conference audience is likely to know very little of what you
%   are going to talk about. So *simplify*!
% - In a 20min talk, getting the main ideas across is hard
%   enough. Leave out details, even if it means being less precise than
%   you think necessary.
% - If you omit details that are vital to the proof/implementation,
%   just say so once. Everybody will be happy with that.
\section{Variable Types and Attributes}
\begin{frame}[t]{Anatomy of a Variable}
\begin{columns}[t]
    \column{0.5\textwidth}
    \begin{itemize}
        \item Variables carry {\bf attributes}.
        \item Attributes are {\bf bound} to variables.
        \item Attributes that are statically bound are typically only stored at compile time.
    \end{itemize}
    
    \column{0.5\textwidth}
    \begin{block}{Common Variable Attributes}
    \begin{itemize}
        \item Name
        \item Address
        \item Type
        \item Value
    \end{itemize}
    \end{block}
\end{columns}
\end{frame}

\begin{frame}{Variable Types}
    \begin{itemize}
        \item A variable's \textbf{type} is an attribute that describes its range of values and operations.
        \item The variable type system of a language reveals a lot about its semantic behavior.
        \item Languages should check for type errors and/or perform \textbf{type coercion} in operations.
    \end{itemize}
\end{frame}

\begin{frame}[fragile]{Type Coercion Examples}
    \begin{itemize}
        \item C/C++
        \begin{itemize}
            \item $double + int \rightarrow double$
            \item $double * int \rightarrow double$
            \item $double / int \rightarrow double$
        \end{itemize}
        \item Javascript
        \begin{itemize}
            \item $string + numeric \rightarrow string$
        \end{itemize}
    \end{itemize}
\end{frame}

\begin{frame}{Variable Typing Throughout History}
\begin{itemize}
    \item Early machine code languages provided no typing.
    \item In the 1950's, FORTRAN provided integer, real, and character typing.
    \begin{itemize}
        \item Name Based Types: I, J, K, L, M, N are integers. All others real.
        \item Explicit typing to override name typing.
    \end{itemize}
    \item In the 1960's, ALGOL introduced fully declarative typing and user-defined types.
    \item COBOL implemented typing by image.
    \item 1970's, Strong typing with enumerated types and record types.
    \item 1980's to present saw the rise of user-defined abstract types.
\end{itemize}
\end{frame}

\begin{frame}{Strong Variable Typing}
\begin{block}{Definition}
\textbf{Strong Typing} is when a language reports all type errors.
\end{block}
\begin{itemize}
    \item Most languages are not fully strongly typed.
    \item C/C++ tend toward strong typing, however they have union types, which are not checked.
    \item Strong typing exists on a spectrum.
\end{itemize}
\end{frame}

\section{Common Primitive Types}
\begin{frame}{Numeric Types}
    \begin{description}
        \item[Integer] - Stores a whole/integral number. Usually supported directly by hardware. Represented internally as a raw binary number. Can be signed or unsigned.
        \item[Floating Point] - Usually represented using the IEEE 754 representation. These are often supported by hardware, but sometimes implemented in software. They have a high degree of error.
        \item[Decimal] - Fixed precision decimal number. Usually represented as BCD (Binary Coded Decimal). These are exact with no round-off error.
        \item[Boolean] - Usually represented by a single byte, with 0 representing false and other values representing true.
    \end{description}
\end{frame}

\begin{frame}{Character Types}
    \begin{description}
        \item[Character] - A single character. Often this is how a programming language represents an individual byte.
        \item[String] - A collection of characters. Sometimes additional attributes are stored.
    \end{description}
\end{frame}

\begin{frame}{Reference Types}
    \begin{description}
        \item[Pointer] - A variable which stores the address of a value in memory. Often has type attributes associated with it.
        \item[Reference] - An abstraction of a pointer. Provides an alias to an actual variable, or it can provide reference to a complex type (as in Java)
    \end{description}
\end{frame}

\begin{frame}[fragile]{Enumerated Types}
   \begin{itemize}
       \item Types with a fixed range of labeled values.
       \item For example, in C:
       \begin{code}
enum day {MONDAY, TUESDAY, WEDNESDAY, THURSDAY, FRIDAY, SATURDAY, SUNDAY};
enum day d = TUESDAY;
       \end{code}
       \item Usually implemented as integers, internally.
       \item Some languages perform error checking on the values.
   \end{itemize} 
\end{frame}

\begin{frame}{Non-Scalar Types}
    \begin{itemize}
        \item Non-Scalar types store lists and groups of information.
        \item \textbf{Arrays} - Homogeneous sequence of items.
        \item \textbf{Associate Arrays} - Associates a key/value pair. Usually heterogeneous.
        \item \textbf{Lists} - Usually internally a doubly linked list, can be heterogeneous.
        \item \textbf{Tuple} - An immutable list.
    \end{itemize}
\end{frame}

\section{User-Defined Types}
\begin{frame}{User-Defined Types}
    \begin{itemize}
        \item First popularized in the ALGOL languages.
        \item Data types derived through combinations of primitive types.
    \end{itemize}
\end{frame}


\begin{frame}[fragile]{Record Types}
    \begin{itemize}
        \item A record is formed out of several fields.
        \item Example: structs in C/C++:\newline
        \begin{code}
struct point {
    double x;
    double y;
};
struct point p;
p.x = 0;
p.y = 0;
        \end{code}
        \item A record type has all of its fields present at all times. Its size is at least as big as the sum of the sizes of its fields.
    \end{itemize}
\end{frame}

\begin{frame}[fragile]{Union Types}
    \begin{itemize}
        \item A union is a data structure that can take on one of several types.
        \item Defined much like a record, but only one field contains data at a given point in time.
        \item Example: union in C \newline
        \begin{code}
union eval_result {
    int i;
    double d;
    float f;
    char c;
    void *ptr;
};
        \end{code}
    \end{itemize}
\end{frame}

\begin{frame}{Abstract Types}
    \begin{itemize}
        \item Abstract types include relationships between types.
        \item The most common system of abstract types is Object Oriented Programming.
    \end{itemize}
\end{frame}

\begin{frame}{Object Oriented Programming}
    \begin{block}{Object}
    An \textbf{object} is any entity with state and behavior. 
    \end{block}
    
    Elements of Object Oriented Programming
    \begin{description}
        \item[Abstraction] - Objects act like black boxes. Details are hidden.
        \item[Encapsulation] - An object maintains its own internal state. It carries all necessary data within itself.
        \item[Inheritance] - An object can be generalized by another object. 
        \item[Polymorphism] - Code can be written to the most general case. An object can be referenced by any time from which it descends.
    \end{description}
\end{frame}

\begin{frame}{Reading and Reference}
    \begin{itemize}
        \item Read Chapter 6 -- Data Types 
        \item IEEE 754 Format: \href{https://www.geeksforgeeks.org/ieee-standard-754-floating-point-numbers/amp/}{https://www.geeksforgeeks.org/ieee-standard-754-floating-point-numbers/amp/}
        \item BCD Format: \href{https://www.geeksforgeeks.org/bcd-or-binary-coded-decimal/}{https://www.geeksforgeeks.org/bcd-or-binary-coded-decimal/}
    \end{itemize}
\end{frame}
\end{document}