% Uncomment for handout
\def\HANDOUT{}


\ifdefined\HANDOUT
\documentclass[handout]{beamer}
\usepackage{pgfpages}
\pgfpagesuselayout{4 on 1}[letterpaper,landscape,border shrink=5mm]
\else
\documentclass{beamer}
\fi

\mode<presentation>
{
  \usetheme{Warsaw}
  \definecolor{sered}{rgb}{0.78, 0.06, 0.18}
  \definecolor{richblack}{rgb}{0.0, 0.0, 0.0}
  \setbeamercolor{structure}{fg=sered,bg=richblack}
  %\setbeamercovered{transparent}
}


\usepackage[english]{babel}
\usepackage[latin1]{inputenc}
\usepackage{times}
\usepackage[T1]{fontenc}
\usepackage{tikz}
\usepackage{graphicx}
\usepackage[export]{adjustbox}
\usepackage{fancyvrb}
\usepackage{amsmath}
\usepackage{amssymb}
\usepackage{esvect}

\makeatletter
\newcommand{\imagesource}[1]{{\centering\hfill\break\hbox{\scriptsize Image Source:\thinspace{\tiny\itshape #1}}\par}}
\newcommand{\image}[3][\@nil]{%
        \def\tmp{#1}%
        \begin{center}
        \ifx\tmp\@nnil
            \includegraphics[max height = 0.55\textheight, max width = \textwidth]{images/#2}
        \else
            \includegraphics[max height = 0.50\textheight, max width = \textwidth]{images/#2}
            \linebreak
            #1
        \fi
        \linebreak
        {\tiny Image Source:\thinspace{\tiny #3}}
        \end{center}
}

\newenvironment{code}{%
 \VerbatimEnvironment
 \begin{adjustbox}{max width=\textwidth, max height=0.7\textheight}
 \begin{BVerbatim}
  }{
  \end{BVerbatim}
 \end{adjustbox}
}

\title{Describing Semantic Behavior}

\author{Robert Lowe}

\institute[Southeast Missouri State University] % (optional, but mostly needed)
{
  Department of Computer Science\\
  Southeast Missouri State University
}

\date[]{}
\subject{}

\pgfdeclareimage[height=1.0cm]{university-logo}{images/semo-logo}
\logo{\pgfuseimage{university-logo}}



\AtBeginSection[]
{
  \begin{frame}<beamer>{Outline}
    \tableofcontents[currentsection]
  \end{frame}
}


\begin{document}

\begin{frame}
  \titlepage
\end{frame}

\begin{frame}{Outline}
  \tableofcontents
\end{frame}


% Structuring a talk is a difficult task and the following structure
% may not be suitable. Here are some rules that apply for this
% solution: 

% - Exactly two or three sections (other than the summary).
% - At *most* three subsections per section.
% - Talk about 30s to 2min per frame. So there should be between about
%   15 and 30 frames, all told.

% - A conference audience is likely to know very little of what you
%   are going to talk about. So *simplify*!
% - In a 20min talk, getting the main ideas across is hard
%   enough. Leave out details, even if it means being less precise than
%   you think necessary.
% - If you omit details that are vital to the proof/implementation,
%   just say so once. Everybody will be happy with that.

\section {Wrapping up BNF}

\begin{frame}[fragile]{Is there a problem here?}
\begin{code}
< Start >      ::= < Expression >

< Expression > ::= < Expression > "+" < Term > 
                   | < Expression > "-" < Term >
                   | < Term >

< Term >       ::= < Term > "*" < Number >
                   | < Term > "/" < Number >
                   | < Number >

< Number >     ::= < Number > < Digit > 
                   | < Digit >

< Digit >      ::= "0" | "1"
\end{code}
\end{frame}

\begin{frame}[fragile]{Actually, NO! }
\begin{columns}
\column{0.6\textwidth}
We can repeat operations without ambiguity:

\begin{code}

< Expression > + < Term >
     |
[ < Expression > + < Term > ] + < Term >
      |
[[ < Term > ] + < Term >]] + < Term > 
      |
[[ < Number > ] + < Term >]] + < Term > 
                     |
[[ < Number > ] + < Number >]] + < Term > 
                                    |
[[ < Number > ] + < Number >]] + < Number > 
      
\end{code}

\column{0.3\textwidth}
\begin{block}{Rules}
\begin{code}
< Expression > ::= 
   < Expression > "+" < Term > 
   | < Expression > "-" < Term >
   | < Term >

< Term > ::= 
   < Term > "*" < Number >
   | < Term > "/" < Number >
   | < Number >
\end{code}
\end{block}
\end{columns}
\end{frame}


\begin{frame}[fragile]{Defining a simplified C-Like Language}
\begin{columns}
\column{0.3\textwidth}
\begin{code}
int main() 
{
    int x;
    real r;
    
    x = input();
    while( x > 0 ) {
        r = r + x;
    }
    
    print(r/2);
}
\end{code}

\column{0.7\textwidth}
\begin{block}{Programs and Functions}
\begin{code}
< Program > ::= 
    < Function-Def >
    | <  Program > < Function-Def >
    
< Function-Def > ::=
    < Signature > < Block >

< Signature > ::=
    < Type > < Identifier > "(" ")"
    | < Type > < Identifier > "(" < Params > ")"
    
< Params > ::=
    < Params > "," < Decl >
    | < Decl>
    
< Block > ::= 
    "{" < Statement-List > "}"
    | "{" "}"

< Statement-List > ::=
    < Statement-List > < Statement >
    | < Statement >
\end{code}
\end{block}
\end{columns}
\end{frame}



\begin{frame}[fragile]{Defining a simplified C-Like Language (Ctd.)}
\begin{columns}
\column{0.3\textwidth}
\begin{code}
int main() 
{
    int x;
    real r;
    
    x = input();
    while( x > 0 ) {
        r = r + x;
    }
    
    print(r/2);
}
\end{code}

\column{0.7\textwidth}
\begin{block}{ Statements }
\begin{code}
< Statement > ::= 
    < Decl > ";" | < Assign > ";" | < Expr > ";" 
    | < Call > ";" | < While > 

< Decl > ::= 
    < Type > < Identifier >
    
< Type > :: = "real" | "int"
    
< Assign > ::=
    < Identifier > "=" < Expr > 
    
< Call > ::=
    < Identifier > "(" ")"
    | < Identifier > "(" < Args > ")" 

< Args > ::=
    < Args > "," < Expr >
    | < Expr >
    
< While > ::=
    "while" "(" < Expr > ")" < Statement >
    | "while" "(" < Expr > ")" < Block >
\end{code}
\end{block}
\end{columns}
\end{frame}


\begin{frame}[fragile]{Defining a simplified C-Like Language (Ctd.)}
\begin{columns}
\column{0.3\textwidth}
\begin{code}
int main() 
{
    int x;
    real r;
    
    x = input();
    while( x > 0 ) {
        r = r + x;
    }
    
    print(r/2);
}
\end{code}

\column{0.7\textwidth}
\begin{block}{ Expressions }
\begin{code}
< Expr > ::= < Expr > "<" < Sum > 
             | < Sum >

< Sum >  ::= < Sum > "+" < Mul >
             | < Sum > "-" < Mul >
             | < Mul >

< Mul >  ::= < Mul > "*" < Value >
             | < Mul > "/" < Value >
             | < Value >
             
< Value > ::= < Number >
             | < Identifier >
             | < Call >
             
< Identifier > ::= < Identifier > < Letter >
                   | < Letter >
\end{code}
\end{block}
\end{columns}
\end{frame}


\begin{frame}[fragile]{BNF Definition of BNF's Syntax}
\begin{code}
<syntax>         ::= <rule> | <rule> <syntax>
<rule>           ::= <opt-whitespace> "<" <rule-name> ">" <opt-whitespace> "::=" <opt-whitespace> <expression> <line-end>
<opt-whitespace> ::= " " <opt-whitespace> | ""
<expression>     ::= <list> | <list> <opt-whitespace> "|" <opt-whitespace> <expression>
<line-end>       ::= <opt-whitespace> <EOL> | <line-end> <line-end>
<list>           ::= <term> | <term> <opt-whitespace> <list>
<term>           ::= <literal> | "<" <rule-name> ">"
<literal>        ::= '"' <text1> '"' | "'" <text2> "'"
<text1>          ::= "" | <character1> <text1>
<text2>          ::= '' | <character2> <text2>
<character>      ::= <letter> | <digit> | <symbol>
<letter>         ::= "A" | "B" | "C" | "D" | "E"  | "F" | "G" | "H" | "I" | "J"  | "K" | "L" | "M" | "N" | "O" 
                     | "P" | "Q" | "R" | "S" | "T"  | "U" | "V" | "W" | "X" | "Y" | "Z" 
                     | "a" | "b" | "c" | "d" | "e"  | "f" | "g" | "h" | "i" | "j"  | "k" | "l" | "m" | "n" | "o" 
                     | "p" | "q" | "r" | "s" | "t"  | "u" | "v" | "w" | "x" | "y" | "z"
<digit>          ::= "0" | "1" | "2" | "3" | "4" | "5" | "6" | "7" | "8" | "9"
<symbol>         ::=  "|" | " " | "!" | "#" | "$"  | "%" | "&" | "(" | ")" | "*"  | "+" | "," | "-" | "." | "/" 
                      | ":" | ";" | ">" | "=" | "<"  | "?" | "@" | "[" | "\" | "]"  | "^" | "_" | "`" | "{" | "}" | "~"
<character1>     ::= <character> | "'"
<character2>     ::= <character> | '"'
<rule-name>      ::= <letter> | <rule-name> <rule-char>
<rule-char>      ::= <letter> | <digit> | "-"
\end{code}
    
\end{frame}
\section {Describing Semantic Behavior}

\begin{frame}{Semantic and Syntactic Domain}
    \begin{itemize}
        \item {\bf Syntactic Domain} -- Form of a language.
        \item {\bf Semantic Domain} -- Meaning/Behavior of a language.
    \end{itemize}
\end{frame}


\begin{frame}[fragile]{Informal Semantics}
    \begin{itemize}
        \item Write a description for the behavior of logical sentential forms.
        \item For example:
        \begin{code}
<While> ::= "while" "(" < Expr > ")" < Block >

If the expression is true, execute the block and then repeat.
    NOTE: True is defined as any non-zero value.
        \end{code}
    \end{itemize}
\end{frame}

\begin{frame}{Formal Semantics}
    \begin{itemize}
        \item There also exists formalisms to describe semantics. One example is Hoare logic.
        \item $p \{S\} q$ is a formal statement that when $p$ is true, executing $S$ will cause $q$ to be true.
        \item Example:
        \begin{enumerate}
            \item Let $p$ be the statement \texttt{< Expr >} is true.
            \item $p \{$ \texttt{< Block >} $\} p$ implies repetition.
            \item $p \{$ \texttt{< Block >} $\} \neg p$ halts.
            \item $\neg p$ skips block entirely.
        \end{enumerate}
        \item Such formalisms are really only done if we wish to prove a mathematical property of a language.
    \end{itemize}
\end{frame}

\begin{frame}{Reading Assignment}
    \begin{itemize}
        \item Finish your reading of Chapter 3
        \item Read \textit{An Axiomatic Basis for Computer Programming} by C.A.R. Hoare (Hoare69.pdf)
    \end{itemize}
\end{frame}

\end{document}
