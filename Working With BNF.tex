% Uncomment for handout
\def\HANDOUT{}


\ifdefined\HANDOUT
\documentclass[handout]{beamer}
\usepackage{pgfpages}
\pgfpagesuselayout{4 on 1}[letterpaper,landscape,border shrink=5mm]
\else
\documentclass{beamer}
\fi

\mode<presentation>
{
  \usetheme{Warsaw}
  \definecolor{sered}{rgb}{0.78, 0.06, 0.18}
  \definecolor{richblack}{rgb}{0.0, 0.0, 0.0}
  \setbeamercolor{structure}{fg=sered,bg=richblack}
  %\setbeamercovered{transparent}
}


\usepackage[english]{babel}
\usepackage[latin1]{inputenc}
\usepackage{times}
\usepackage[T1]{fontenc}
\usepackage{tikz}
\usepackage{graphicx}
\usepackage[export]{adjustbox}
\usepackage{fancyvrb}
\usepackage{amsmath}
\usepackage{amssymb}
\usepackage{esvect}

\makeatletter
\newcommand{\imagesource}[1]{{\centering\hfill\break\hbox{\scriptsize Image Source:\thinspace{\tiny\itshape #1}}\par}}
\newcommand{\image}[3][\@nil]{%
        \def\tmp{#1}%
        \begin{center}
        \ifx\tmp\@nnil
            \includegraphics[max height = 0.55\textheight, max width = \textwidth]{images/#2}
        \else
            \includegraphics[max height = 0.50\textheight, max width = \textwidth]{images/#2}
            \linebreak
            #1
        \fi
        \linebreak
        {\tiny Image Source:\thinspace{\tiny #3}}
        \end{center}
}

\newenvironment{code}{%
 \VerbatimEnvironment
 \begin{adjustbox}{max width=\textwidth, max height=0.7\textheight}
 \begin{BVerbatim}
  }{
  \end{BVerbatim}
 \end{adjustbox}
}

\title{Working with BNF}


\author{Robert Lowe}

\institute[Southeast Missouri State University] % (optional, but mostly needed)
{
  Department of Computer Science\\
  Southeast Missouri State University
}

\date[]{}
\subject{}

\pgfdeclareimage[height=1.0cm]{university-logo}{images/semo-logo}
\logo{\pgfuseimage{university-logo}}



\AtBeginSection[]
{
  \begin{frame}<beamer>{Outline}
    \tableofcontents[currentsection]
  \end{frame}
}


\begin{document}

\begin{frame}
  \titlepage
\end{frame}

\begin{frame}{Outline}
  \tableofcontents
\end{frame}


% Structuring a talk is a difficult task and the following structure
% may not be suitable. Here are some rules that apply for this
% solution: 

% - Exactly two or three sections (other than the summary).
% - At *most* three subsections per section.
% - Talk about 30s to 2min per frame. So there should be between about
%   15 and 30 frames, all told.

% - A conference audience is likely to know very little of what you
%   are going to talk about. So *simplify*!
% - In a 20min talk, getting the main ideas across is hard
%   enough. Leave out details, even if it means being less precise than
%   you think necessary.
% - If you omit details that are vital to the proof/implementation,
%   just say so once. Everybody will be happy with that.

\section{Evaluating Grammars}

\begin{frame}{The Problem with Ambiguity}
\begin{itemize}
    \item Recall that a grammar is meant to be an exact specification of a language.
    \item Compilers and interpreters will {\bf parse} input based on the grammar.
    \item If there is any ambiguity in a grammar, we cannot precisely specify what a compiler should do!
\end{itemize}
\end{frame}

\begin{frame}[fragile]{Our Grammar So Far}
\begin{code}
< Start >      ::= < Expression >

< Expression > ::= < Expression > "+" < Expression > 
                   | < Expression > "-" < Expression >
                   | < Expression > "*" < Expression >
                   | < Expression > "/" < Expression >
                   | < Number >

< Number >     ::= < Number > < Digit > 
                   | < Digit >

< Digit >      ::= "0" | "1"
\end{code}
\end{frame}

\begin{frame}[fragile]{Ambiguous Rules}
\begin{columns}
\column{0.5\textwidth}
\begin{code}
11+10*11

< Expression > + < Expression >
[11] + [10*11]
[11] + [[10]*[11]]

--- or ---

< Expression > * < Expression >
[11+10] * [11]
[[11]+[10]] * [11]

\end{code}
\column{0.5\textwidth}
\begin{block}{An Ambiguous Rule}
\begin{code}
< Expression > ::= 
    < Expression > "+" < Expression > 
    | < Expression > "-" < Expression >
    | < Expression > "*" < Expression >
    | < Expression > "/" < Expression >
    | < Number >
\end{code}
\end{block}
\end{columns}
\end{frame}

\begin{frame}[fragile]{Parse Trees}
\begin{columns}
\column{0.5\textwidth}
\begin{block}{Parse Tree 1}
\begin{code}

[11] + [[10]*[11]]

         +
       /   \
     11     *
          /   \
        10     11  
        
\end{code}
\end{block}
\column{0.5\textwidth}
\begin{block}{Parse Tree 2}
\begin{code}

[[11]+[10]] * [11]

             *
           /   \
          +     11
        /   \
       11   10 
       
\end{code}
\end{block}
\end{columns}    
\end{frame}

\begin{frame}[fragile]{Evaluating Parse Trees}
\begin{columns}
\column{0.5\textwidth}
\begin{block}{Parse Tree 1}
\begin{code}
(1)
         +
       /   \
     11     *
          /   \
        10     11  
        
(2)        
         +
       /   \
     11    110  
        
(3)        
       1001 

\end{code}
\end{block}
\column{0.5\textwidth}
\begin{block}{Parse Tree 2}
\begin{code}
(1)
             *
           /   \
          +     11
        /   \
       11   10 
       
(2)
             *
           /   \
         101    11
       
(3)
           1111

\end{code}
\end{block}
\end{columns}    
\end{frame}

\section{Correcting Problems in Grammars}

\begin{frame}{Recursion and Ambiguity}
\begin{block}{Ambiguous Grammar}
    A grammar is {\bf ambiguous} if it contains at least one rule which can produce more than one parse tree.
\end{block}
\begin{itemize}
    \item {\bf Left Recursion} is when a rule recurses to the left of a non terminal. \texttt{< A > ::= < A > "b"}
    \item {\bf Right Recursion} is when a rule recurses to the right of a non terminal. \texttt{< A > ::= "b" < A > }
    \item {\bf Multiple Recursion} is any mixture of left or right recursion. \texttt{< A > :: = < A > "b" < A >}
    \item Multiple recursion is a potential source of ambiguity!
\end{itemize}
\end{frame}

\begin{frame}{Correcting Ambiguity}
\begin{itemize}
    \item There can never be an algorithm for determining whether a context-free algorithm is ambiguous.
    \item We can prevent ambiguity by:
    \begin{itemize}
        \item Convention 
        \item By Controlling Recursion
    \end{itemize}
\end{itemize}
\end{frame}

\begin{frame}[fragile]{A corrected Grammar}
\begin{code}
< Start >      ::= < Expression >

< Expression > ::= < Expression > "+" < Term > 
                   | < Expression > "-" < Term >
                   | < Term >

< Term >       ::= < Term > "*" < Number >
                   | < Term > "/" < Number >
                   | < Number >

< Number >     ::= < Number > < Digit > 
                   | < Digit >

< Digit >      ::= "0" | "1"
\end{code}
\end{frame}

\begin{frame}[fragile]{Applying the Corrected Grammar}
\begin{code}
11+10*11

< Expression > "+" < Term >
       |              | 
   < Term >        < Term > "*" < Number >
       |              |             |
   < Number >     < Number >    < Number >
       |              |             |
      11        + [  10      *      11 ]
                      
\end{code}
\end{frame}


\begin{frame}{A Word on the Order of Operations}
    \begin{itemize}
        \item The recursion depth of an operator determines its {\bf precedence}. Deeper operators are processed first.
        \item The recursion side of an operator determines its {\bf associativity}. Left recursion is used for left associativity, right recursion for right associativity.
        \item There is still ambiguity in a "+" and "-", but associativity breaks the tie if we assume we expand the left most non-terminal first.
    \end{itemize}
\end{frame}


\end{document}
