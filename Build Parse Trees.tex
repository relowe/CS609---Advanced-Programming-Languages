% Uncomment for handout
\def\HANDOUT{}


\ifdefined\HANDOUT
\documentclass[handout]{beamer}
\usepackage{pgfpages}
\pgfpagesuselayout{4 on 1}[letterpaper,landscape,border shrink=5mm]
\else
\documentclass{beamer}
\fi

\mode<presentation>
{
  \usetheme{Warsaw}
  \definecolor{sered}{rgb}{0.78, 0.06, 0.18}
  \definecolor{richblack}{rgb}{0.0, 0.0, 0.0}
  \setbeamercolor{structure}{fg=sered,bg=richblack}
  %\setbeamercovered{transparent}
}


\usepackage[english]{babel}
\usepackage[latin1]{inputenc}
\usepackage{times}
\usepackage[T1]{fontenc}
\usepackage{tikz}
\usepackage{graphicx}
\usepackage[export]{adjustbox}
\usepackage{fancyvrb}
\usepackage{amsmath}
\usepackage{amssymb}
\usepackage{esvect}

\makeatletter
\newcommand{\imagesource}[1]{{\centering\hfill\break\hbox{\scriptsize Image Source:\thinspace{\tiny\itshape #1}}\par}}
\newcommand{\image}[3][\@nil]{%
        \def\tmp{#1}%
        \begin{center}
        \ifx\tmp\@nnil
            \includegraphics[max height = 0.55\textheight, max width = \textwidth]{images/#2}
        \else
            \includegraphics[max height = 0.50\textheight, max width = \textwidth]{images/#2}
            \linebreak
            #1
        \fi
        \linebreak
        {\tiny Image Source:\thinspace{\tiny #3}}
        \end{center}
}

\newenvironment{code}{%
 \VerbatimEnvironment
 \begin{adjustbox}{max width=\textwidth, max height=0.7\textheight}
 \begin{BVerbatim}
  }{
  \end{BVerbatim}
 \end{adjustbox}
}


\newenvironment{scaled}{%
 \begin{adjustbox}{max width=\textwidth, max height=0.7\textheight}
 }{
 \end{adjustbox}
}

\title{Building Parse Trees}


\author{Robert Lowe}

\institute[Southeast Missouri State University] % (optional, but mostly needed)
{
  Department of Computer Science\\
  Southeast Missouri State University
}

\date[]{}
\subject{}

\pgfdeclareimage[height=1.0cm]{university-logo}{images/semo-logo}
\logo{\pgfuseimage{university-logo}}



\AtBeginSection[]
{
  \begin{frame}<beamer>{Outline}
    \tableofcontents[currentsection]
  \end{frame}
}


\begin{document}

\begin{frame}
  \titlepage
\end{frame}

\begin{frame}{Outline}
  \tableofcontents
\end{frame}


% Structuring a talk is a difficult task and the following structure
% may not be suitable. Here are some rules that apply for this
% solution: 

% - Exactly two or three sections (other than the summary).
% - At *most* three subsections per section.
% - Talk about 30s to 2min per frame. So there should be between about
%   15 and 30 frames, all told.

% - A conference audience is likely to know very little of what you
%   are going to talk about. So *simplify*!
% - In a 20min talk, getting the main ideas across is hard
%   enough. Leave out details, even if it means being less precise than
%   you think necessary.
% - If you omit details that are vital to the proof/implementation,
%   just say so once. Everybody will be happy with that.

\section{Planning for Parse Trees}
\begin{frame}{Parse Tree Basics}
    \begin{itemize}
        \item Each node in a parse tree represents one of two things:
        \begin{item}
            \item An operation.
            \item A value.
        \end{item}
        \item Nodes in a parse tree can have a variable number of children.
        \item Interior nodes represent operators.
        \item Leaves are values.
    \end{itemize}
\end{frame}


\begin{frame}{Identifying Parse Trees in Non-Terminals}
    \begin{itemize}
        \item Each rule which results in an operation should form a parse tree.
        \item Not all non-terminals will have a corresponding parse tree node.
        \item It helps to sketch out the results of each rule's parse tree.
    \end{itemize} 
\end{frame}


\begin{frame}[fragile]{Calc Parser Grammar}
\begin{code}
< Program >     ::= < Program > < Statement > 
                    | < Statement >

< Statement >   ::= < Expression > NEWLINE

< Expression >  ::= < Term > < Expression' >

< Expression' > ::= PLUS < Term  > < Expression' >
                    | MINUS < Term > < Expression' >
                    | ""
                    
< Term >        ::= < Factor > < Term' >

< Term' >       ::= TIMES  < Factor > < Term' >
                    | DIVIDE < Factor > < Term' >
                    | ""

< Factor >      ::= < Base > POW < Factor >
                    | < Base >

< Base >        ::= LPAREN < Expression > RPAREN
                    | MINUS < Expression > 
                    | < Number >

< Number >      ::= INTLIT
                    | REALLIT
\end{code}
\end{frame}


\begin{frame}[fragile]{Program Parse Tree}
\begin{code}
                                    (1)
< Program >     ::= < Program > < Statement > 
                           (1)
                    | < Statement >


                    
\end{code}

\pause

\begin{code}
Parse Tree
                        PROGRAM
                     /      |      \
                   (1) ... (1) ... (1)
\end{code}
\end{frame}


\begin{frame}[fragile]{Statement Parse Tree}
\begin{code}
                         (1)
< Statement >   ::= < Expression > NEWLINE

                    
\end{code}

\pause

\begin{code}
Parse Tree

                        (1)
\end{code}
\end{frame}


\begin{frame}[fragile]{Expression Parse Tree}
\begin{code}
                       (1)
< Expression >  ::= < Term > < Expression' >

                            (2)
< Expression' > ::= PLUS < Term  > < Expression' >
                    | MINUS < Term > < Expression' >
                    | ""

                    
\end{code}

\pause

\begin{code}
Parse Tree

    +              -
  /   \          /   \
(1)   (2)      (1)   (2)
\end{code}
\end{frame}


\section{Coding Parse Trees}
\begin{frame}{Object Oriented Parse Trees}
    A hierarchy of objects:
    \begin{itemize}
        \item \texttt{class ParseTree}
            \begin{itemize}
                \item \texttt{class Number: public ParseTree}
                \item \texttt{class UnaryOp : public ParseTree}
                \begin{itemize}
                    \item \texttt{class Neg: public UnaryOp }
                \end{itemize}
                \item \texttt{class BinaryOp : public ParseTree}
                \begin{itemize}
                    \item \texttt{class Add : public BinaryOp}
                    \item \texttt{class Sub : public BinaryOp}
                    \item \texttt{class Mul : public BinaryOp}
                    \item \texttt{class Div : public BinaryOp}
                    \item \texttt{class Pow : public BinaryOp}
                \end{itemize}
                \item \texttt{class NaryOp : public ParseTree}
                \begin{itemize}
                    \item \texttt{class Program : NaryOp}
                \end{itemize}
            \end{itemize}
        \end{itemize}
\end{frame}


\begin{frame}[fragile]{Building the Program Parse Tree}
\begin{code}
/*
 * < Program >     ::= < Program > < Statement > 
 *                      | < Statement >
 */
ParseTree *Parser::parse_program()
{
    Program *result = new Program(curtok());

    // Technically, this is not LL(1), but it is easy enough to handle 
    // this with a while loop
    while(not has(TEOF)) {
        result->push(parse_statement());
    }
    return result;
}
\end{code}
\end{frame}

\begin{frame}[fragile]{Building the Statement Parse Tree}
\begin{code}
/*
 * < Statement >   ::= < Expression > NEWLINE
 */
ParseTree *Parser::parse_statement()
{
    ParseTree *result = parse_expression();
    must_be(NEWLINE);
    next();

    return result;
}
\end{code}
\end{frame}


\begin{frame}[fragile]{Building the Expression Parse Tree}
\begin{code}
/*
 * < Expression >  ::= < Term > < Expression' >
 */
ParseTree *Parser::parse_expression()
{
    ParseTree *left = parse_term();
    return parse_expression_prime(left);
}
\end{code}
\end{frame}

\begin{frame}[t,fragile]{Building the Expression Parse Tree}

\begin{code}
/*
 * < Expression' > ::= PLUS < Term  > < Expression' >
 *                     | MINUS < Term > < Expression' >
 *                     | ""
 */
\end{code}
\begin{columns}[T]
\column{0.5\textwidth}
\begin{code}
ParseTree *Parser::parse_expression_prime(ParseTree *left)
{
    if(has(PLUS)) {
        // start the parse tree
        Add *result = new Add(curtok());
        next();

        //get the children
        result->left(left);
        result->right(parse_term());
        return parse_expression_prime(result);
\end{code}
\column{0.5\textwidth}
\begin{code}
    } else if(has(MINUS)) {
        // start the parse tree
        Sub *result = new Sub(curtok());
        next();

        // get the children
        result->left(left);
        result->right(parse_term());
        return parse_expression_prime(result);
    }

    // nothing to do for the empty case
    return left;
}
\end{code}
\end{columns}
\end{frame}

\begin{frame}{Suggested Exercises}
    \begin{itemize}
        \item Sketch the parse tree formation for the rest of the rules in calc.
        \item Think about how you would implement those.
        \item Read the modified parser code as well as the parse tree code.
        \item Watch the video tutorials on pointers and tree data structures.
    \end{itemize}
\end{frame}

\end{document}
