% Uncomment for handout
\def\HANDOUT{}


\ifdefined\HANDOUT
\documentclass[handout]{beamer}
\usepackage{pgfpages}
\pgfpagesuselayout{4 on 1}[letterpaper,landscape,border shrink=5mm]
\else
\documentclass{beamer}
\fi

\mode<presentation>
{
  \usetheme{Warsaw}
  \definecolor{sered}{rgb}{0.78, 0.06, 0.18}
  \definecolor{richblack}{rgb}{0.0, 0.0, 0.0}
  \setbeamercolor{structure}{fg=sered,bg=richblack}
  %\setbeamercovered{transparent}
}


\usepackage[english]{babel}
\usepackage[latin1]{inputenc}
\usepackage{times}
\usepackage[T1]{fontenc}
\usepackage{tikz}
\usepackage{graphicx}
\usepackage[export]{adjustbox}
\usepackage{fancyvrb}
\usepackage{amsmath}
\usepackage{amssymb}
\usepackage{esvect}

\makeatletter
\newcommand{\imagesource}[1]{{\centering\hfill\break\hbox{\scriptsize Image Source:\thinspace{\tiny\itshape #1}}\par}}
\newcommand{\image}[3][\@nil]{%
        \def\tmp{#1}%
        \begin{center}
        \ifx\tmp\@nnil
            \includegraphics[max height = 0.55\textheight, max width = \textwidth]{images/#2}
        \else
            \includegraphics[max height = 0.50\textheight, max width = \textwidth]{images/#2}
            \linebreak
            #1
        \fi
        \linebreak
        {\tiny Image Source:\thinspace{\tiny #3}}
        \end{center}
}

\newenvironment{code}{%
 \VerbatimEnvironment
 \begin{adjustbox}{max width=\textwidth, max height=0.7\textheight}
 \begin{BVerbatim}
  }{
  \end{BVerbatim}
 \end{adjustbox}
}


\newenvironment{scaled}{%
 \begin{adjustbox}{max width=\textwidth, max height=0.7\textheight}
 }{
 \end{adjustbox}
}

\title{Building a Lexer}


\author{Robert Lowe}

\institute[Southeast Missouri State University] % (optional, but mostly needed)
{
  Department of Computer Science\\
  Southeast Missouri State University
}

\date[]{}
\subject{}

\pgfdeclareimage[height=1.0cm]{university-logo}{images/semo-logo}
\logo{\pgfuseimage{university-logo}}



\AtBeginSection[]
{
  \begin{frame}<beamer>{Outline}
    \tableofcontents[currentsection]
  \end{frame}
}


\begin{document}

\begin{frame}
  \titlepage
\end{frame}

% Structuring a talk is a difficult task and the following structure
% may not be suitable. Here are some rules that apply for this
% solution: 

% - Exactly two or three sections (other than the summary).
% - At *most* three subsections per section.
% - Talk about 30s to 2min per frame. So there should be between about
%   15 and 30 frames, all told.

% - A conference audience is likely to know very little of what you
%   are going to talk about. So *simplify*!
% - In a 20min talk, getting the main ideas across is hard
%   enough. Leave out details, even if it means being less precise than
%   you think necessary.
% - If you omit details that are vital to the proof/implementation,
%   just say so once. Everybody will be happy with that.
\begin{frame}{Phase 1 --- Lexer Framework}
    \begin{enumerate}
        \item Construct the Tokens class with 2 tokens, \texttt{INVALID} and \texttt{EOF}
        \item Construct the \texttt{Lexeme} named tuple.
        \item Create the \texttt{Lexer} class with the following methods:
        \begin{itemize}
            \item \texttt{read}
            \item \texttt{consume}
            \item \texttt{skip\_space}
        \end{itemize}
        \item Construct empty versions of the \texttt{group1}, \texttt{group2}, \texttt{group3} methods which return \texttt{False}
        \item Construct the \texttt{next} method.
        \item Construct the unit test for the lexer, and verify that it produces a stream of \texttt{INVALID} tokens terminating with an \texttt{EOF} token.
    \end{enumerate}
\end{frame}

\begin{frame}{Phase 2 --- Tokens and Group 1}
    \begin{enumerate}
        \item Define all the tokens.
        \item Implement the \texttt{group1} function fully.
        \begin{itemize}
            \item The lexer functions should update \texttt{cur\_tok} on a match.
            \item The lexer functions return \texttt{True} on match, \texttt{False} otherwise.
            \item Lexer functions should call \texttt{consume} to consume all characters in their matched lexeme.
            \item List the tokens as tuples of the form \texttt{(lex, Token)}
        \end{itemize}
        \item Test that the lexer identifies the group 1 tokens.
    \end{enumerate}
\end{frame}

\begin{frame}[fragile]{Phase 3 --- Group 2 Tokens}
    \begin{block}{Maintaining Possible Tokens}
    \begin{code}
remain = lambda s : [ tok for tok in tokens if tok[0].startswith(s) ]
tokens = remain(s) 
    \end{code}
    \end{block}
    \begin{enumerate}
        \item Create the token list.
        \item Use the above lambda to filter possible tokens.
        \item Return the matching token and/or invalid.
        \item Verify that the group 2 tokens are matched.
    \end{enumerate}
\end{frame}

\begin{frame}{Phase 4 --- Group 3 Tokens}
    \begin{enumerate}
        \item Subdivide the group 3 tokens into what they start with. Usually separating into letter and number groups.
        \item Create functions for each subdivision.
        \begin{itemize}
            \item These functions will always return \texttt{True}
            \item If they fail to match, they should set the \texttt{cur\_tok} field to invalid.
        \end{itemize}
        \item Call the appropriate function based on the contents of \texttt{cur\_char} when \texttt{group3} is called.
        \item Verify that all tokens get matched.
        \item Insert an invalid symbol and verify that invalid tokens are still matched.
    \end{enumerate}
\end{frame}

\end{document}