% Uncomment for handout
\def\HANDOUT{}


\ifdefined\HANDOUT
\documentclass[handout]{beamer}
\usepackage{pgfpages}
\pgfpagesuselayout{4 on 1}[letterpaper,landscape,border shrink=5mm]
\else
\documentclass{beamer}
\fi

\mode<presentation>
{
  \usetheme{Warsaw}
  \definecolor{sered}{rgb}{0.78, 0.06, 0.18}
  \definecolor{richblack}{rgb}{0.0, 0.0, 0.0}
  \setbeamercolor{structure}{fg=sered,bg=richblack}
  %\setbeamercovered{transparent}
}


\usepackage[english]{babel}
\usepackage[latin1]{inputenc}
\usepackage{times}
\usepackage[T1]{fontenc}
\usepackage{tikz}
\usepackage{graphicx}
\usepackage[export]{adjustbox}
\usepackage{fancyvrb}
\usepackage{amsmath}
\usepackage{amssymb}
\usepackage{esvect}

\makeatletter
\newcommand{\imagesource}[1]{{\centering\hfill\break\hbox{\scriptsize Image Source:\thinspace{\tiny\itshape #1}}\par}}
\newcommand{\image}[3][\@nil]{%
        \def\tmp{#1}%
        \begin{center}
        \ifx\tmp\@nnil
            \includegraphics[max height = 0.55\textheight, max width = \textwidth]{images/#2}
        \else
            \includegraphics[max height = 0.50\textheight, max width = \textwidth]{images/#2}
            \linebreak
            #1
        \fi
        \linebreak
        {\tiny Image Source:\thinspace{\tiny #3}}
        \end{center}
}

\newenvironment{code}{%
 \VerbatimEnvironment
 \begin{adjustbox}{max width=\textwidth, max height=0.7\textheight}
 \begin{BVerbatim}
  }{
  \end{BVerbatim}
 \end{adjustbox}
}


\newenvironment{scaled}{%
 \begin{adjustbox}{max width=\textwidth, max height=0.7\textheight}
 }{
 \end{adjustbox}
}

\title{Building a Tree-Walk Interpreter}


\author{Robert Lowe}

\institute[Southeast Missouri State University] % (optional, but mostly needed)
{
  Department of Computer Science\\
  Southeast Missouri State University
}

\date[]{}
\subject{}

\pgfdeclareimage[height=1.0cm]{university-logo}{images/semo-logo}
\logo{\pgfuseimage{university-logo}}



\AtBeginSection[]
{
  \begin{frame}<beamer>{Outline}
    \tableofcontents[currentsection]
  \end{frame}
}


\begin{document}

\begin{frame}
  \titlepage
\end{frame}

\begin{frame}{Outline}
  \tableofcontents
\end{frame}


% Structuring a talk is a difficult task and the following structure
% may not be suitable. Here are some rules that apply for this
% solution: 

% - Exactly two or three sections (other than the summary).
% - At *most* three subsections per section.
% - Talk about 30s to 2min per frame. So there should be between about
%   15 and 30 frames, all told.

% - A conference audience is likely to know very little of what you
%   are going to talk about. So *simplify*!
% - In a 20min talk, getting the main ideas across is hard
%   enough. Leave out details, even if it means being less precise than
%   you think necessary.
% - If you omit details that are vital to the proof/implementation,
%   just say so once. Everybody will be happy with that.

\section{Basic Plan}
\begin{frame}{Elements of a Tree-Walk Interpreter}
    \begin{itemize}
        \item A mechanism for returning values from evaluation.
        \item A mutually recursive evaluation function for each parse tree node.
    \end{itemize}
\end{frame}

\begin{frame}[fragile]{Returning Results}
\begin{columns}
\column{0.5\textwidth}
\begin{itemize}
    \item C++ is a \textbf{statically typed} language.
    \item Expressions can return one of several results.
    \item A \texttt{union} is one way to account for this.
\end{itemize}
\column{0.5\textwidth}
\begin{code}
union ResultField
{
    int i;
    double r;
};
\end{code}
\end{columns}
\end{frame}

\begin{frame}[fragile]{Returning Results}
\begin{columns}
\column{0.5\textwidth}
\begin{itemize}
    \item We need some way to identify which \texttt{union} field to access.
    \item Pairing a \texttt{union} with an \texttt{enum} is a common approach.
\end{itemize}
\column{0.5\textwidth}
\begin{code}
enum ResultType 
{
    VOID=0,
    INTEGER,
    REAL
};

struct Result
{
    // the value and type of the result
    ResultField val;
    ResultType type;
};
\end{code}
\end{columns}
\end{frame}


\begin{frame}[fragile]{Convenience Methods and Macros}
\begin{code}
// convert result types to strings
extern const char* RTSTR[];

// print result values
std::ostream& operator<<(std::ostream& os, const Result &result);

// A macro to extract the numeric result from Result
#define NUM_RESULT(res) ((res).type == INTEGER ? (res).val.i : (res).val.r)

// A macro to assign the correct numeric field
#define NUM_ASSIGN(res, n) ((res).type == INTEGER ? (res).val.i=(n) : (res).val.r=(n))
\end{code}
\begin{itemize}
    \item Retrieving and assigning numbers is a common operation.
    \item Printing a result is frequently needed.
    \item For debugging, we may want to print the types.
\end{itemize}
\end{frame}


\begin{frame}[fragile]{Parse Tree Evaluation Functions}
\begin{columns}
\column{0.5\textwidth}
\begin{itemize}
    \item We add a \textbf{pure virtual} function to the \texttt{ParseTree} class.
    \item Each type of \texttt{ParseTree} node will have its own mutually recursive \texttt{eval} function.
    \item \texttt{eval} will return a \texttt{Result} object for each operation.
\end{itemize}
\column{0.5\textwidth}
\begin{code}
class ParseTree
{
public:
    //constructor and destructor
    ParseTree(LexerToken &token);
    virtual ~ParseTree();
    ...
    // evaluate the parse tree
    virtual Result eval()=0;
    ...
private:
    ...
};
\end{code}
\end{columns}
\end{frame}

\section{Implementation}
\begin{frame}[fragile]{Evaluating Numbers}
\begin{code}
Number::Number(LexerToken _token) : ParseTree(_token)
{
    //get the number's value
    if(_token == INTLIT) {
        _val.type = INTEGER;
        _val.val.i = stoi(_token.lexeme);
    } else if(_token == REALLIT) {
        _val.type = REAL;
        _val.val.r = stod(_token.lexeme);
    }
}

Result Number::eval()
{
    return _val;
}
\end{code}
\end{frame}


\begin{frame}[fragile]{Evaluate Unary Operations}
\begin{code}
Result Neg::eval()
{
    //eval the child and then negate it
    Result result = child()->eval();
    NUM_ASSIGN(result, -NUM_RESULT(result));

    return result;
}
\end{code}
\end{frame}


\begin{frame}[fragile]{Evaluate Binary Operations}
\begin{code}
Result Add::eval() 
{
    // evaluate the children
    Result l = left()->eval();
    Result r = right()->eval();

    // get the type of the result
    Result result;
    result.type = coerce(l, r);

    // perform the operation
    NUM_ASSIGN(result, NUM_RESULT(l) + NUM_RESULT(r));

    return result;
}
\end{code}
\end{frame}


\begin{frame}[fragile]{Type Coercion}
\begin{code}
static ResultType coerce(Result left, Result right) 
{
    // if the types match, there is no coercion
    if(left.type == right.type) return left.type;

    // if either left or right is void, so is the result
    if(left.type == VOID or right.type == VOID) return VOID;

    // perform type widening
    if((left.type == REAL and right.type == INTEGER) or 
       (left.type == INTEGER and right.type == REAL)) {
        return REAL;
    }

    //TODO: Technically, if we make it here we have an error. For now, we will
    //      just default to void. Eventually, we should report a type error.
    return VOID;
}
\end{code}
\end{frame}


\begin{frame}[fragile]{Evaluating Programs}
\begin{code}
Result Program::eval()
{
    // evaluate and print each statement in the program
    for(auto itr = begin(); itr != end(); itr++) {
        std::cout << (*itr)->eval() << std::endl;
    }

    // programs return void
    Result result;
    result.type = VOID;

    return result;
}
\end{code}
\end{frame}

\begin{frame}{Suggested Exercises}
    \begin{itemize}
        \item Compile and run the complete calc interpreter.
        \item Try out a series of expressions in REPL mode. (Running without a file name.)
        \item Does the output match expectations? What seems odd?
        \item Where is the problem? Is it a language or implementation bug?
        \item Study the main function in \texttt{calc.cpp}. How does the REPL loop work? Why was that necessary; that is, why not just pass \texttt{cin} into the lexer and run the program?
    \end{itemize}
\end{frame}

\end{document}

