% Uncomment for handout
\def\HANDOUT{}


\ifdefined\HANDOUT
\documentclass[handout]{beamer}
\usepackage{pgfpages}
\pgfpagesuselayout{4 on 1}[letterpaper,landscape,border shrink=5mm]
\else
\documentclass{beamer}
\fi

\mode<presentation>
{
  \usetheme{Warsaw}
  \definecolor{sered}{rgb}{0.78, 0.06, 0.18}
  \definecolor{richblack}{rgb}{0.0, 0.0, 0.0}
  \setbeamercolor{structure}{fg=sered,bg=richblack}
  %\setbeamercovered{transparent}
}


\usepackage[english]{babel}
\usepackage[latin1]{inputenc}
\usepackage{times}
\usepackage[T1]{fontenc}
\usepackage{tikz}
\usepackage{graphicx}
\usepackage[export]{adjustbox}
\usepackage{fancyvrb}
\usepackage{amsmath}
\usepackage{amssymb}
\usepackage{esvect}

\makeatletter
\newcommand{\imagesource}[1]{{\centering\hfill\break\hbox{\scriptsize Image Source:\thinspace{\tiny\itshape #1}}\par}}
\newcommand{\image}[3][\@nil]{%
        \def\tmp{#1}%
        \begin{center}
        \ifx\tmp\@nnil
            \includegraphics[max height = 0.55\textheight, max width = \textwidth]{images/#2}
        \else
            \includegraphics[max height = 0.50\textheight, max width = \textwidth]{images/#2}
            \linebreak
            #1
        \fi
        \linebreak
        {\tiny Image Source:\thinspace{\tiny #3}}
        \end{center}
}

\newenvironment{code}{%
 \VerbatimEnvironment
 \begin{adjustbox}{max width=\textwidth, max height=0.7\textheight}
 \begin{BVerbatim}
  }{
  \end{BVerbatim}
 \end{adjustbox}
}


\newenvironment{scaled}{%
 \begin{adjustbox}{max width=\textwidth, max height=0.7\textheight}
 }{
 \end{adjustbox}
}

\title{Implementing Variables}


\author{Robert Lowe}

\institute[Southeast Missouri State University] % (optional, but mostly needed)
{
  Department of Computer Science\\
  Southeast Missouri State University
}

\date[]{}
\subject{}

\pgfdeclareimage[height=1.0cm]{university-logo}{images/semo-logo}
\logo{\pgfuseimage{university-logo}}



\AtBeginSection[]
{
  \begin{frame}<beamer>{Outline}
    \tableofcontents[currentsection]
  \end{frame}
}


\begin{document}

\begin{frame}
  \titlepage
\end{frame}

\begin{frame}{Outline}
  \tableofcontents
\end{frame}


% Structuring a talk is a difficult task and the following structure
% may not be suitable. Here are some rules that apply for this
% solution: 

% - Exactly two or three sections (other than the summary).
% - At *most* three subsections per section.
% - Talk about 30s to 2min per frame. So there should be between about
%   15 and 30 frames, all told.

% - A conference audience is likely to know very little of what you
%   are going to talk about. So *simplify*!
% - In a 20min talk, getting the main ideas across is hard
%   enough. Leave out details, even if it means being less precise than
%   you think necessary.
% - If you omit details that are vital to the proof/implementation,
%   just say so once. Everybody will be happy with that.
\section{Handling Symbols}

\begin{frame}{Overview}
    \begin{itemize}
        \item Create a Data Structure to store symbols.
        \item Implement situations where symbols are added to the data structure.
        \item Implement binding of attributes to symbols.
        \item Implement retrieval of attributes from symbols.
    \end{itemize}
\end{frame}

\begin{frame}{Symbol Table Basics}
    \begin{itemize}
        \item A \textbf{Symbol Table} maps names onto collections of attributes.
        \item Usually implemented using some sort of associative data structure.
        \item Error checking is often performed at insertion and retrieval times.
    \end{itemize}
\end{frame}


\begin{frame}[fragile]{The Reference Environment Object}
    \begin{code}
class RefEnv { 
 public: 
     // constructor 
     RefEnv(); 
  
     // declare a variable 
     virtual void declare(const std::string &name, ResultType type); 
  
     // check to see if a name exists in the environment 
     virtual bool exists(const std::string &name); 
  
     // retrieve a variable associative array style 
     virtual Result& operator[](const std::string &name); 
  
 private: 
     std::map<std::string, Result> _symtab; 
 };
    \end{code}
\end{frame}

\begin{frame}[fragile]{Declaring Symbols}
    \begin{code}
// declare a variable 
void RefEnv::declare(const std::string &name, ResultType type) 
{ 
     // names must be unique 
     if(exists(name)) { 
         throw std::runtime_error("Redeclaration of " + name); 
     } 
  
     // create the variable and add it to the table 
     Result var; 
     var.type = type; 
     _symtab[name] = var; 
}
    \end{code}
\end{frame}

\begin{frame}[fragile]{Retrieving Symbols}
    \begin{code}
// retrieve a variable associative array style 
 Result& RefEnv::operator[](const std::string &name) 
 { 
     // names must exist 
     if(not exists(name)) { 
         throw std::runtime_error(name + " not defined."); 
     } 
  
     return _symtab[name]; 
 }
    \end{code}
\end{frame}

\section{Implementing Variables}
\begin{frame}[fragile]{Implementing Declaration}
    \begin{code}
Result VarDecl::eval() 
 { 
     ResultType var_type; 
     Result result; 
     result.type = VOID; 
  
     //get the variable type 
     switch(token().token) 
     { 
         case INTEGER_DECL: 
             var_type = INTEGER; 
             break; 
         case REAL_DECL: 
             var_type = REAL; 
             break; 
     } 
  
     //perform the declaration 
     env.declare(child()->token().lexeme, var_type); 
  
     return result; 
 }
    \end{code}
\end{frame}


\begin{frame}[fragile]{Implementing Assignment}
    \begin{code}
Result Assign::eval() 
{ 
     // get the value and name to assign 
     Result val = right()->eval(); 
     std::string name = left()->token().lexeme; 
  
     //perform the assignment 
     NUM_ASSIGN(env[name], NUM_RESULT(val)); 
  
     Result result; 
     result.type = VOID; 
  
     return result; 
}
    \end{code}
\end{frame}


\begin{frame}[fragile]{Implementing Value Retrieval}
    \begin{code}
Result Var::eval() 
{ 
     return env[token().lexeme];; 
}
    \end{code}
\end{frame}


\begin{frame}[fragile]{Future Directions}
    \begin{itemize}
        \item Right now, we have a single global symbol table:
        \begin{code}
// global reference environment for variables 
 static RefEnv env;
        \end{code}
        \item We will soon work with multiple environments.
        \item We will also soon provide a mechanism to nest environments to implement variable scoping.
        \item Flexible environments will allow us to implement object oriented and functional features!
    \end{itemize}
\end{frame}

\end{document}

