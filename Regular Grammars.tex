% Uncomment for handout
%\def\HANDOUT{}


\ifdefined\HANDOUT
\documentclass[handout]{beamer}
\usepackage{pgfpages}
\pgfpagesuselayout{4 on 1}[letterpaper,landscape,border shrink=5mm]
\else
\documentclass{beamer}
\fi

\mode<presentation>
{
  \usetheme{Warsaw}
  \definecolor{sered}{rgb}{0.78, 0.06, 0.18}
  \definecolor{richblack}{rgb}{0.0, 0.0, 0.0}
  \setbeamercolor{structure}{fg=sered,bg=richblack}
  %\setbeamercovered{transparent}
}


\usepackage[english]{babel}
\usepackage[latin1]{inputenc}
\usepackage{times}
\usepackage[T1]{fontenc}
\usepackage{tikz}
\usepackage{graphicx}
\usepackage[export]{adjustbox}
\usepackage{fancyvrb}
\usepackage{amsmath}
\usepackage{amssymb}
\usepackage{esvect}
\usepackage{smartdiagram}

\usetikzlibrary{arrows,automata,positioning,calc}

\makeatletter
\newcommand{\imagesource}[1]{{\centering\hfill\break\hbox{\scriptsize Image Source:\thinspace{\tiny\itshape #1}}\par}}
\newcommand{\image}[3][\@nil]{%
        \def\tmp{#1}%
        \begin{center}
        \ifx\tmp\@nnil
            \includegraphics[max height = 0.55\textheight, max width = \textwidth]{images/#2}
        \else
            \includegraphics[max height = 0.50\textheight, max width = \textwidth]{images/#2}
            \linebreak
            #1
        \fi
        \linebreak
        {\tiny Image Source:\thinspace{\tiny #3}}
        \end{center}
}

\newenvironment{code}{%
 \VerbatimEnvironment
 \begin{adjustbox}{max width=\textwidth, max height=0.7\textheight}
 \begin{BVerbatim}
  }{
  \end{BVerbatim}
 \end{adjustbox}
}

\newenvironment{scaled}{%
 \begin{adjustbox}{max width=\textwidth, max height=0.7\textheight}
 }{
 \end{adjustbox}
}

\title{Regular Grammars}


\author{Robert Lowe}

\institute[Southeast Missouri State University] % (optional, but mostly needed)
{
  Department of Computer Science\\
  Southeast Missouri State University
}

\date[]{}
\subject{}

\pgfdeclareimage[height=1.0cm]{university-logo}{images/semo-logo}
\logo{\pgfuseimage{university-logo}}



\AtBeginSection[]
{
  \begin{frame}<beamer>{Outline}
    \tableofcontents[currentsection]
  \end{frame}
}


\begin{document}

\begin{frame}
  \titlepage
\end{frame}

\begin{frame}{Outline}
  \tableofcontents
\end{frame}


% Structuring a talk is a difficult task and the following structure
% may not be suitable. Here are some rules that apply for this
% solution: 

% - Exactly two or three sections (other than the summary).
% - At *most* three subsections per section.
% - Talk about 30s to 2min per frame. So there should be between about
%   15 and 30 frames, all told.

% - A conference audience is likely to know very little of what you
%   are going to talk about. So *simplify*!
% - In a 20min talk, getting the main ideas across is hard
%   enough. Leave out details, even if it means being less precise than
%   you think necessary.
% - If you omit details that are vital to the proof/implementation,
%   just say so once. Everybody will be happy with that.
\section {Regular Grammars}
\begin{frame}{Left and Right Regular Grammars}
\begin{columns}
\column{0.5\textwidth}
\begin{block}{Left Regular Grammar}
    \begin{align*}
        A & \rightarrow a \\
        A & \rightarrow Ba
    \end{align*}
\end{block}

\column{0.5\textwidth}
\begin{block}{Right Regular Grammar}
    \begin{align*}
        A & \rightarrow a \\
        A & \rightarrow aB
    \end{align*}
\end{block}

\end{columns}

\vspace{1cm}
{\bf Definitions}
\begin{itemize}
    \item $A,B$ --- non-terminal
    \item $a$ --- terminal
\end{itemize}
\end{frame}


\begin{frame}{Beware Mixed Grammars}
\begin{columns}
\column{0.5\textwidth}
{\bf Regular}
\begin{align*}
    A & \rightarrow aB \\
    B & \rightarrow Cb \\
    C & \rightarrow cC \\
    C & \rightarrow c
\end{align*}
\column{0.5\textwidth}
{\bf Context Free}
\begin{align*}
    S & \rightarrow aA \\
    A & \rightarrow Sb \\
    S & \rightarrow \epsilon
\end{align*}
\end{columns}
\end{frame}


\begin{frame}{Example Regular Languages}
\begin{enumerate}
    \item $L = \{ac^nb : n \geq 1\}$
    \item $L = \{d^m | d^m.d^n : m \geq 1, n\geq1, d=0\ldots 9\}$
    \item $L = \{\mathrm{while} | \mathrm{int} | \mathrm{real} | letter^n : n\geq 1\}$ 
\end{enumerate}
\end{frame}


\begin{frame}{Regular Expressions}
\begin{itemize}
    \item Character Matching
    \begin{itemize}
        \item Literal Character, matches itself
        \item \texttt{\bf .} Wildcard, matches any character
    \end{itemize}
    \item Quantifiers (Applies to Preceding Character or Group) 
    \begin{itemize}
        \item \texttt{\bf *} Zero or More
        \item \texttt{\bf +} One or More
        \item \texttt{\bf ?} Zero or One
    \end{itemize}
    \item Grouping
    \begin{itemize}
        \item \texttt{\bf ( )} Group
        \item \texttt{\bf |} Or
    \end{itemize}
    \item Common Abbreviations
    \begin{itemize}
        \item \texttt{\bf [A-Z,a-z,0-9]} Character Ranges
    \end{itemize}
\end{itemize}

\end{frame}


\begin{frame}[fragile]{Example Regular Expressions}
\begin{enumerate}
    \item $L = \{ac^nb : n \geq 1\}$ \newline
    \begin{code}
ac+b
    \end{code}
    \vspace{1cm}
    \item $L = \{d^m | d^m.d^n : m \geq 1, n\geq1, d=0\ldots 9\}$ \newline
    \begin{code}
[0-9]+|[0-9]+\.[0-9]+
    \end{code}
    \vspace{1cm}
    \item $L = \{\mathrm{while} | \mathrm{int} | \mathrm{real} | letter^n : n\geq 1\}$ \newline
    \begin{code}
while|int|real|[a-zA-Z]+
    \end{code}
\end{enumerate}
\end{frame}


\section {Finite State Automatons}
\begin{frame}{Recognizing a Regular Language}
    \begin{itemize}
        \item All regular languages can be recognized by a finite state automaton.
        \item A finite state automaton scans its input character by character and transitions among states.
        \item If the automaton stops in an accepting state, the string matches. Otherwise it does not. 
    \end{itemize}
\end{frame}


\begin{frame}{Finite State Automata}
\[
M = \langle Q, \Sigma, \delta, q_0, F\rangle
\]
\begin{description}
\item[$Q$] Finite Set of States
\item[$\Sigma$] Finite Alphabet
\item[$\delta$] Transition Function $Q \times \Sigma \rightarrow Q$ 
\item[$q_0$] Initial State $\q_0 \in Q$
\item[$F$] Set of Final/Accepting State $f\in Q$
\end{description}
\end{frame}


\begin{frame}{Example Finite State Automaton}
\[
L=\{ac^nb:n \geq 1\}
\]
\begin{align*}
    Q = & \{q_0, q_1, q_2, q_3, q_4\} \\
    \Sigma = & \{a, b, c \} \\
    \delta = & \{ ((q_0, a), q_1),  ((q_0, b), q_4), ((q_0, c), q_4), \\
             & \ \ ((q_1, a), q_4),  ((q_1, b), q_4), ((q_1, c), q_2), \\
             & \ \ ((q_2, a), q_4),  ((q_2, b), q_3), ((q_2, c), q_2) \\
             & \ \ ((q_3, a), q_4),  ((q_3, b), q_4), ((q_3, c), q_4) \\
             & \ \ ((q_4, a), q_4),  ((q_4, b), q_4), ((q_4, c), q_4) \\
             &\} \\
    q_0 = & q_0 \\
    F = & \{ q_3 \}
\end{align*}
\end{frame}

\begin{frame}{Graphical Finite State Automaton}
\[
L=\{ac^nb:n \geq 1\}
\]
\begin{center}
\begin{scaled}
\begin{tikzpicture}[node distance=2cm,auto]
    \node[state, initial] (q0) {$q_0$};
    \node[state] (q1) [right of=q0] {$q_1$};
    \node[state] (q2) [right of=q1] {$q_2$};
    \node[state,accepting] (q3) [right of=q2] {$q_3$};
    \node[state] (q4) [below = of $(q1)!0.5!(q2)$] {$q_4$};
    
    \path[->]
    (q0) edge [bend left] node {\texttt{a}} (q1)
    (q1) edge [bend left] node {\texttt{c}} (q2)
    (q2) edge [loop above] node {\texttt{c}} ( )
    (q2) edge [bend left] node {\texttt{b}} (q3)
    (q0) edge [bend right] node {\texttt{b,c}} (q4)
    (q1) edge [bend right] node {\texttt{a,b}} (q4)
    (q2) edge [bend left] node {\texttt{a,b}} (q4)
    (q3) edge [bend left] node {\texttt{a,b,c}} (q4)
    (q4) edge [loop below] node {\texttt{a,b,c}} ()
    
\end{tikzpicture}
\end{scaled}
\end{center}
\end{frame}

\begin{frame}{Abbreviated Graphical Finite State Automaton}
    \[
    L=\{ac^nb:n \geq 1\}
    \]
    \begin{center}
    \begin{scaled}
    \begin{tikzpicture}[node distance=2cm,auto]
        \node[state, initial] (q0) {$q_0$};
        \node[state] (q1) [right of=q0] {$q_1$};
        \node[state] (q2) [right of=q1] {$q_2$};
        \node[state,accepting] (q3) [right of=q2] {$q_3$};
        
        \path[->]
        (q0) edge [bend left] node {\texttt{a}} (q1)
        (q1) edge [bend left] node {\texttt{c}} (q2)
        (q2) edge [loop above] node {\texttt{c}} ( )
        (q2) edge [bend left] node {\texttt{b}} (q3)
        
    \end{tikzpicture}
    \end{scaled}
    
    \vspace{1cm}
    We usually only show valid transitions and just assume a universal ``rejection'' state!
    \end{center}
\end{frame}

\begin{frame}{Number Grammar}
    \[
    L = \{d^m | d^m.d^n : m \geq 1, n\geq1, d=0\ldots 9\}
    \]
    
    \begin{center}
    \begin{scaled}
    \begin{tikzpicture}[node distance=2cm,auto]
        \node[state,initial] (q0) {$q_0$};
        \node[state,accepting] (I) [right of=q0]{$I$};
        \node[state] (q1) [below of=I] {$q_1$};
        \node[state,accepting] (R) [right of=q1] {$R$};
        
        \path[->]
        (q0) edge [bend left] node {$0\ldots 9$} (I)
        (I) edge [loop above] node {$0\ldots 9$} ( )
        (I) edge [bend right] node {\texttt{.}} (q1)
        (q1) edge [bend left] node {$0\ldots9$} (R)
        (R) edge [loop right] node {$0\ldots 9$} ()
    \end{tikzpicture}
    \end{scaled}
    \end{center}
\end{frame}

\begin{frame}{Keyword Literal Grammar}
    \[
        L = \{\mathrm{while} | \mathrm{int} | \mathrm{real} | letter^n : n\geq 1\}
    \]
    \begin{center}
    \begin{adjustbox}{max width=\textwidth, max height=0.55\textheight}
    \begin{tikzpicture}[node distance=2cm,auto]
        \node[state,initial] (q0) {$q_0$};
        \node[state] (w0) [above right of=q0] {$w_0$};
        \node[state] (i0) [above left of=q0] {$i_0$};
        \node[state,accepting] (L) [below of=q0] {$L$};
        \node[state] (w1) [right of=w0] {$w_1$};
        \node[state] (w2) [right of=w1] {$w_2$};
        \node[state] (w3) [right of=w2] {$w_3$};
        \node[state,accepting] (W) [right of=w3] {$W$};
        \node[state] (i1) [left of=i0] {$i_1$};
        \node[state,accepting] (I) [left of=i1] {$I$};
        \node[state] (r1) [below left of=L] {$r_1$};
        \node[state] (r0) [left of=r1] {$r_0$};
        \node[state] (r2) [right of=r1] {$r_2$};
        \node[state,accepting] (R) [right of=r2]{$R$};
        
        \path[->]
        (q0) edge [bend left] node {\texttt{w}} (w0)
        (w0) edge [bend left] node {\texttt{h}} (w1)
        (w1) edge [bend left] node {\texttt{i}} (w2)
        (w2) edge [bend left] node {\texttt{l}} (w3)
        (w3) edge [bend left] node {\texttt{e}} (W)
        
        (q0) edge [bend right] node {\texttt{i}} (i0)
        (i0) edge [bend left] node {\texttt{n}} (i1)
        (i1) edge [bend left] node {\texttt{t}} (I)
        
        (q0) edge [bend right] node {\texttt{r}} (r0)
        (r0) edge [bend left] node {\texttt{e}} (r1)
        (r1) edge [bend left] node {\texttt{a}} (r2)
        (r2) edge [bend left] node {\texttt{l}} (R)
        
        (q0) edge [bend right] node {$*$} (L)
        (w0) edge [bend left] node {$*$} (L)
        (w1) edge [bend left] node {$*$} (L)
        (w2) edge [bend left] node {$*$} (L)
        (w3) edge [bend left] node {$*$} (L)
        (W) edge [bend left] node {$*$} (L)
        (i0) edge [bend right] node {$*$} (L)
        (i1) edge [bend right] node {$*$} (L)
        (I) edge [bend right] node {$*$} (L)
        (r0) edge [bend left] node {$*$} (L)
        (r1) edge [bend left] node {$*$} (L)
        (r2) edge [bend right] node {$*$} (L)
        (R) edge [bend right] node {$*$} (L)
    \end{tikzpicture}
    \end{adjustbox}
    \end{center}
\end{frame}


\section {Lexical Analysis and Parsing}
\begin{frame}{Processing a Programming Language}
\smartdiagramset{back arrow disabled=true}
\smartdiagram[flow diagram:horizontal]{Source, Lexer, Parser, Processor}

\vspace{1cm}
\begin{enumerate}
    \item The source code is read by the lexical analyzer or {\bf lexer}.
    \item The lexer recognizes the regular portion of the grammar, producing a stream of {\bf tokens}.
    \item The {\bf parser} reads the tokens and recognizes the context free portion of the grammar.
    \item The {\bf processor} processes the data structure built by the parser.
\end{enumerate}
\end{frame}

\begin{frame}{Dividing the Grammar}
    \begin{enumerate}
        \item Extract all the terminal strings and give them token labels.
        \item Replace terminal strings with token labels.
        \item Extract all regular rules and give them token labels.
        \item Replace all regular rules with token labels.
        \item Repeat steps 3 and 4 taking care that the lexer grammar is still regular.
    \end{enumerate}
\end{frame}

\begin{frame}{Reading Assignment}
    Read Chapters 4.1 -- 4.2
\end{frame}

\end{document}
