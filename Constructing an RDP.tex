% Uncomment for handout
\def\HANDOUT{}


\ifdefined\HANDOUT
\documentclass[handout]{beamer}
\usepackage{pgfpages}
\pgfpagesuselayout{4 on 1}[letterpaper,landscape,border shrink=5mm]
\else
\documentclass{beamer}
\fi

\mode<presentation>
{
  \usetheme{Warsaw}
  \definecolor{sered}{rgb}{0.78, 0.06, 0.18}
  \definecolor{richblack}{rgb}{0.0, 0.0, 0.0}
  \setbeamercolor{structure}{fg=sered,bg=richblack}
  %\setbeamercovered{transparent}
}


\usepackage[english]{babel}
\usepackage[latin1]{inputenc}
\usepackage{times}
\usepackage[T1]{fontenc}
\usepackage{tikz}
\usepackage{graphicx}
\usepackage[export]{adjustbox}
\usepackage{fancyvrb}
\usepackage{amsmath}
\usepackage{amssymb}
\usepackage{esvect}

\makeatletter
\newcommand{\imagesource}[1]{{\centering\hfill\break\hbox{\scriptsize Image Source:\thinspace{\tiny\itshape #1}}\par}}
\newcommand{\image}[3][\@nil]{%
        \def\tmp{#1}%
        \begin{center}
        \ifx\tmp\@nnil
            \includegraphics[max height = 0.55\textheight, max width = \textwidth]{images/#2}
        \else
            \includegraphics[max height = 0.50\textheight, max width = \textwidth]{images/#2}
            \linebreak
            #1
        \fi
        \linebreak
        {\tiny Image Source:\thinspace{\tiny #3}}
        \end{center}
}

\newenvironment{code}{%
 \VerbatimEnvironment
 \begin{adjustbox}{max width=\textwidth, max height=0.7\textheight}
 \begin{BVerbatim}
  }{
  \end{BVerbatim}
 \end{adjustbox}
}


\newenvironment{scaled}{%
 \begin{adjustbox}{max width=\textwidth, max height=0.7\textheight}
 }{
 \end{adjustbox}
}

\title{Constructing and Recursive Descent Parser}


\author{Robert Lowe}

\institute[Southeast Missouri State University] % (optional, but mostly needed)
{
  Department of Computer Science\\
  Southeast Missouri State University
}

\date[]{}
\subject{}

\pgfdeclareimage[height=1.0cm]{university-logo}{images/semo-logo}
\logo{\pgfuseimage{university-logo}}



\AtBeginSection[]
{
  \begin{frame}<beamer>{Outline}
    \tableofcontents[currentsection]
  \end{frame}
}


\begin{document}

\begin{frame}
  \titlepage
\end{frame}

\begin{frame}{Outline}
  \tableofcontents
\end{frame}


% Structuring a talk is a difficult task and the following structure
% may not be suitable. Here are some rules that apply for this
% solution: 

% - Exactly two or three sections (other than the summary).
% - At *most* three subsections per section.
% - Talk about 30s to 2min per frame. So there should be between about
%   15 and 30 frames, all told.

% - A conference audience is likely to know very little of what you
%   are going to talk about. So *simplify*!
% - In a 20min talk, getting the main ideas across is hard
%   enough. Leave out details, even if it means being less precise than
%   you think necessary.
% - If you omit details that are vital to the proof/implementation,
%   just say so once. Everybody will be happy with that.
\section{Preparing the Grammar}
\begin{frame}{Basic Steps}
    \begin{enumerate}
        \item Remove Left Recursion
        \item Compute First and Follow Sets
        \item Restructure Rules to be LL(1)
    \end{enumerate}
\end{frame}

\begin{frame}[t,fragile]{Remove Left Recursion -- Part 1}
\begin{columns}[t]
\column{0.5\textwidth}
{\tiny\bf Original}

\begin{code}
< Program > ::= 
    < Function-Def >
    | <  Program > < Function-Def >




< Function-Def > ::= 
    < Signature > < Block >

< Signature > ::= 
    < Type > IDENTIFIER LPAREN RPAREN
    | < Type > IDENTIFIER LPAREN < Params > RPAREN

< Params > ::= 
    < Params > COMMA < Decl >
    | < Decl>
   
   
    
< Block > ::= 
    LBRACE < Statement-List > RBRACE
    | LBRACE RBRACE

< Statement-List > ::= 
    < Statement-List > < Statement >
    | < Statement >
\end{code}
\column{0.5\textwidth}
{\tiny\bf Corrected}

\begin{code}
< Program > ::=
    < Function-Def > < Program' > 

< Program' > ::=
    < Function-Def > < Program' >
    | ""

< Function-Def > ::= 
    < Signature > < Block >

< Signature > ::=
    < Type > IDENTIFIER LPAREN RPAREN
    | < Type > IDENTIFIER LPAREN < Params > RPAREN

< Params > ::= 
    < Decl > < Params' >
    
< Params' > ::= 
    COMMA < Decl > < Params' >
    | ""
    
< Block > ::=
    LBRACE < Statement-List > RBRACE
    | LBRACE RBRACE

< Statement-List > ::= 
    < Statement > < Statement-List' >

< Statement-List' > ::= 
    < Statement > < Statement-List' >
    | ""
\end{code}
\end{columns}
\end{frame}

\begin{frame}[t,fragile]{Remove Left Recursion -- Part 2}
\begin{columns}[t]
\column{0.5\textwidth}
{\bf\tiny Original}

\begin{code}
< Statement > ::= 
    < Decl > SEMI
    | < Assign > SEMI
    | < Expr > SEMI
    | < Call > SEMI
    | < While > 
    | < If > 

< Decl > ::= 
    < Type > IDENTIFIER

< Type > ::=
    REAL | INT

< Assign > ::= 
    IDENTIFIER ASSIGN < Expr > 

< Call > ::= 
    IDENTIFIER LPAREN RPAREN
    | IDENTIFIER LPAREN < Args > RPAREN

< Args > ::=
    < Args > COMMA < Expr >
    | < Expr >




< While > ::= 
    WHILE LPAREN < Expr > RPAREN < Statement >
    | WHILE LPAREN < Expr > RPAREN < Block >

< If > ::= 
    IF LPAREN < Expr > RPAREN < Statement >
    | IF LPAREN < Expr > RPAREN < Block >
\end{code}
\column{0.5\textwidth}
{\bf\tiny Corrected}

\begin{code}
< Statement > ::= 
    < Decl > SEMI
    | < Assign > SEMI
    | < Expr > SEMI
    | < Call > SEMI
    | < While > 
    | < If > 

< Decl > ::=
    < Type > IDENTIFIER

< Type > ::= 
    REAL | INT
    
< Assign > ::= 
    IDENTIFIER ASSIGN < Expr > 

< Call > ::= 
    IDENTIFIER LPAREN RPAREN
    | IDENTIFIER LPAREN < Args > RPAREN

< Args > ::=
    < Expr > < Args' >

< Args' > ::= 
    COMMA < Expr > < Args' >
    | ""

< While > ::= 
    WHILE LPAREN < Expr > RPAREN < Statement >
    | WHILE LPAREN < Expr > RPAREN < Block >

< If > ::= 
    IF LPAREN < Expr > RPAREN < Statement >
    | IF LPAREN < Expr > RPAREN < Block >
\end{code}
\end{columns}
\end{frame}


\begin{frame}[t,fragile]{Remove Left Recursion -- Part 3}
\begin{columns}[t]
\column{0.5\textwidth}
{\bf\tiny Original}

\begin{code}
< Expr > ::= 
    < Expr > LT < Sum > 
    | < Expr > LTE < Sum >
    | < Expr > GT < Sum >
    | < Expr > GTE < Sum >
    | < Expr > EQUAL < Sum >
    | < Sum >




< Sum > ::= < Sum > PLUS < Mul >
    | < Sum > MINUS < Mul >
    | < Mul >





< Mul > ::= 
    < Mul > TIMES < Value >
    | < Mul > DIVIDE < Value >
    | < Value >




< Value > ::= 
    < Number >
    | IDENTIFIER
    | < Call >

< Number > ::= INTNUM | REALNUM
\end{code}
\column{0.5\textwidth}
{\bf\tiny Corrected}

\begin{code}
< Expr > ::= 
    < Sum > < Expr' >

< Expr' > ::= 
    LT < Sum > < Expr' > 
    | LTE < Sum > < Expr' >
    | GT < Sum > < Expr' >
    | GTE < Sum > < Expr' >
    | EQUAL < Sum > < Expr' >
    | ""

< Sum > ::= 
    < Mul > < Sum' >

< Sum' > ::=
    PLUS < Mul > < Sum' >
    | MINUS < Mul > < Sum' >
    | ""

< Mul > ::= 
    < Value > < Mul' > 

< Mul' > ::=
    TIMES < Value > < Mul' >
    | DIVIDE < Value > < Mul' >
    | ""

< Value > ::= 
    < Number >
    | IDENTIFIER
    | < Call >
    | LPAREN < Expr > RPAREN

< Number >::=  INTNUM | REALNUM
\end{code}
\end{columns}
\end{frame}


\begin{frame}[fragile]{Build First and Follow and Fix Problems -- Part 1}
\begin{code}
< Program' >        ::= < Function-Def > < Program' >
                        | ""
  - First(< Function-Def > < Program' >) = First( < Type > ) = { REAL, INT }
  - Follow(< Program' >) = Follow(< Program >) = { EOF }
\end{code}
\end{frame}


\begin{frame}[fragile]{Build First and Follow and Fix Problems -- Part 2}
\begin{code}
< Signature >       ::= < Type > IDENTIFIER LPAREN RPAREN
                        | < Type > IDENTIFIER LPAREN < Params > RPAREN
  - First(< Type > IDENTIFIER LPAREN RPAREN) = First(< Type >)
  - First(< Type > IDENTIFIER LPAREN < Params > RPAREN) = First(< Type >)
  - First(< Type >) = { REAL, INT }
\end{code}
\pause

\begin{code}


Solution
========
< Signature >       ::= < Type > IDENTIFIER LPAREN < Signature' >

< Signature' >      ::= < Params > RPAREN
                        | RPAREN
\end{code}
\end{frame}


\begin{frame}[fragile]{Build First and Follow and Fix Problems -- Part 3}
\begin{code}
< Params' >         ::= COMMA < Decl > < Params' >
                        | ""
  - First(COMMA < Decl > < Params' >) = { COMMA }
  - Follow(< Params' >) = { RPAREN }
\end{code}
\end{frame}


\begin{frame}[fragile]{Build First and Follow and Fix Problems -- Part 4}
\begin{code}
< Block >           ::= LBRACE < Statement-List > RBRACE
                        | LBRACE RBRACE
  - First(LBRACE < Statement-List > RBRACE) = { LBRACE }
  - First(LBRACE RBRACE) = { LBRACE }
\end{code}
\pause
\begin{code}


Solution
========
< Block >           ::= LBRACE < Block' >

< Block' >          ::= RBRACE
                        | < Statement-List > RBRACE
\end{code}
\end{frame}


\begin{frame}[fragile]{Build First and Follow and Fix Problems -- Part 5}
\begin{code}
< Statement-List' > ::= < Statement > < Statement-List' >
                        | ""
  - First(< Statement > < Statement-List' >) = First(Statement)
  - First(< Statement >) = { REAL, INT, IDENTIFIER, INTNUM, REALNUM, LPAREN, WHILE, IF }
  - Follow(< Statement-List' >) = { RBRACE }
\end{code}
\end{frame}


\begin{frame}[fragile]{Build First and Follow and Fix Problems -- Part 6}
\begin{code}
< Statement >       ::= < Decl > SEMI
                        | < Assign > SEMI
                        | < Expr > SEMI
                        | < Call > SEMI
                        | < While > 
                        | < If > 
  - First(< Decl > SEMI) = First(< Decl >) = First(< Type >) = { REAL, INT }
  - First(< Assign > SEMI) = First(< Assign >) = { IDENTIFIER }
  - First(< Expr > SEMI) = First( < Expr > ) = { IDENTIFIER, INTNUM, REALNUM, LPAREN }
  - First(< Call > SEMI) = First(< Call >) = { IDENTIFIER }
  - First(< While >) = First(< While >) = { WHILE }
  - First(< If >) = First(< If >) = { IF }


< Call >            ::= IDENTIFIER LPAREN RPAREN
                        | IDENTIFIER LPAREN < Args > RPAREN
  - First(IDENTIFIER LPAREN RPAREN) = { IDENTIFER }
  - First(IDENTIFIER LPAREN < Args > RPAREN) = { IDENTIFIER }
\end{code}
\end{frame}


\begin{frame}[fragile]{Build First and Follow and Fix Problems -- Part 6}
\begin{code}
Solution
========
< Statement >       ::= < Decl > SEMI
                        | IDENTIFIER < Statement' > SEMI
                        | < While > 
                        | < If > 
                        | < Expr > SEMI


< Statement' >      ::= ASSIGN < Expr >
                        | LPAREN < Call' >
                        | < Expr' >
                        
< Call' >            ::= RPAREN
                        | < Args > RPAREN
\end{code}
\end{frame}

\begin{frame}[fragile]{Build First and Follow and Fix Problems -- Part 7}
\begin{code}
< Args' >           ::= COMMA < Expr > < Args' >
                        | ""
  - First(COMMA < Expr > < Args' >) = { COMMA }
  - Follow(< Args' >) = { RPAREN }
\end{code}
\end{frame}


\begin{frame}[fragile]{Build First and Follow and Fix Problems -- Part 8}
\begin{code}
< While >           ::= WHILE LPAREN < Expr > RPAREN < Statement >
                        | WHILE LPAREN < Expr > RPAREN < Block >
  - First(WHILE LPAREN < Expr > RPAREN < Statement >) = { WHILE }
  - First(WHILE LPAREN < Expr > RPAREN < Block >) = { WHILE }


< If >              ::= IF LPAREN < Expr > RPAREN < Statement >
                        | IF LPAREN < Expr > RPAREN < Block >
  - First(IF LPAREN < Expr > RPAREN < Statement >) = { IF }
  - First(IF LPAREN < Expr > RPAREN < Block >) = { IF }
\end{code}
\end{frame}

\begin{frame}[fragile]{Build First and Follow and Fix Problems -- Part 8}
\begin{code}
Solution
========
< While >           ::= WHILE LPAREN < Expr > RPAREN < Body >


< If >              ::= IF LPAREN < Expr > RPAREN < Body >


< Body >            ::= < Block >
                        | Statement
\end{code}
\end{frame}


\begin{frame}[fragile]{Build First and Follow and Fix Problems -- Part 9}
\begin{code}
< Expr' >           ::= LT < Sum > < Expr' > 
                        | LTE < Sum > < Expr' >
                        | GT < Sum > < Expr' >
                        | GTE < Sum > < Expr' >
                        | EQUAL < Sum > < Expr' >
                        | ""
  - First(LT < Sum > < Expr' >) = { LT }
  - First(LTE < Sum > < Expr' >) = { LTE }
  - First(GT < Sum > < Expr' >) = { GT }
  - First(GTE < Sum > < Expr' >) = { GTE }
  - First(EQUAL < Sum > < Expr' >) = { EQUAL }
  - Follow(< Expr' >) = Follow(< Expr >) = { RPAREN, SEMI, COMMA }
\end{code}
\end{frame}


\begin{frame}[fragile]{Build First and Follow and Fix Problems -- Part 10}
\begin{code}
< Sum' >            ::=  PLUS < Mul > < Sum' >
                        | MINUS < Mul > < Sum' >
                        | ""
  - First(PLUS < Mul > < Sum' >) = { PLUS }
  - First(MINUS < Mul > < Sum' >) = { MINUS }
  - Follow(< Sum' >) = Follow( < Expr' >) = { RPAREN, SEMI, COMMA }


< Mul' >            ::=  TIMES < Value > < Mul' >
                        | DIVIDE < Value > < Mul' >
                        | ""
  - First(TIMES < Value > < Mul' >) = { TIMES }
  - First(DIVIDE < Value > < Mul' >) = { DIVIDE }
  - Follow(< Mul' >) = Follow( < Sum' >) = { RPAREN, SEMI, COMMA }
\end{code}
\end{frame}


\begin{frame}[fragile]{Build First and Follow and Fix Problems -- Part 11}
\begin{code}
< Value >           ::= < Number >
                        | IDENTIFIER
                        | < Call >
                        | LPAREN < Expr > RPAREN
  - First(< Number >) = { INTNUM, REALNUM }
  - First(IDENTIFIER) = { IDENTIFIER }
  - First(< Call >) = { IDENTIFIER }
  - Follow(< Value >) = Follow(< Mul' >) = { RPAREN, SEMI, COMMA }
\end{code}
\pause
\begin{code}


Solution
========
< Value >           ::= < Number >
                        | IDENTIFIER < Value' >
                        | LPAREN < Expr > RPAREN

< Value' >          ::= LPAREN < Call' >
                        | ""
\end{code}
\end{frame}

\section{Constructing the Basic Parser}
\begin{frame}{The Parser Object}
    \begin{itemize}
        \item Public Interface
        \begin{itemize}
            \item A constructor which accepts a \texttt{Lexer} object.
            \item A function called \textt{parse} which returns true on success, false on failure.
        \end{itemize}
        \item Internal State
        \begin{itemize}
            \item \texttt{lexer} -- The token stream
            \item \texttt{errors} -- The number of errors we have experienced during parsing.
        \end{itemize}
    \end{itemize}
\end{frame}


\begin{frame}{Internal Functions}
    \begin{itemize}
        \item \texttt{next} -- Advance the lexer to the next token
        \item \texttt{match} -- Matches a token or list of tokens.
        \item \texttt{have} -- Matches token or list of tokens, on success advances lexer
        \item \texttt{must\_be} -- Attempts to match token(s), prints error on failure. Advances the lexer.
        \item \texttt{parse\_<rule>} -- Mutually recursive functions derived from the bnf.
    \end{itemize}
\end{frame}


\begin{frame}[fragile]{The Initial Class}
\begin{code}
class Parser:
    """
    A recursive descent parser, which uses a token stream from the lexer object.
    """

    def __init__(self, lexer):
        self.lexer = lexer
        self.errors = 0
\end{code}
\end{frame}

\begin{frame}[fragile]{\texttt{next}}
\begin{code}
    def next(self):
        """
        Advance the lexer to the next token.
        """
        lexer.next()
\end{code}
\end{frame}

\begin{frame}[fragile]{\texttt{match}}
\begin{code}
    def match(self, token):
        """
        Attempts to match the current token to token. If token is an iterable
        object, then we attempt to match any of the items in token.
        Returns True on a match, False otherwise.
        """

        # iterable logic
        if hasattr(token, '__iter__'):
            return self.lexer.cur_tok.token in token

        # singular comparison logic
        return self.lexer.cur_tok.token == token
\end{code}
\end{frame}


\begin{frame}[fragile]{\texttt{have}}
\begin{code}
    def have(self, token):
        """
        Returns True and advances the lexer if token is the current token.
        Returns False otherwise.
        """
        if self.match(token):
            self.next()
            return True
        return False
\end{code}
\end{frame}

\begin{frame}[fragile]{\texttt{must\_be}}
\begin{code}
    def must_be(self, token, error_msg="Unexpected Token"):
        """
        Advances the lexer if token is the current token.
        Sets error state to true, advances the lexer, and prints the error
        message otherwise.
        """
        # check for success
        if self.match(token):
            self.next()
            return

        # error!
        self.errors += 1
        error_tok = self.lexer.cur_tok
        self.next()
        print("Error: %s %s:\"%s\" at Line %d Column %d"%(error_msg, 
            error_tok.token, error_tok.lex, error_tok.line, error_tok.col))
\end{code}
\end{frame}

\section{Constructing Production Rules}
\begin{frame}[fragile]{The Basic Idea}
Given something like:

\begin{code}
< A > ::= T1 < B >
         |T2 < C >
         |T3 < A >
         
         
\end{code}

Produce something like:

\begin{code}


def parse_a(self):
    if self.have(Token.T1):
        self.parse_b()
    elif self.have(Token.T2):
        self.parse_c()
    elif self.have(Token.T3):
        self.parse_a()
\end{code}
\end{frame}

\begin{frame}{Nuances and Subtleties}
    \begin{itemize}
        \item Sometimes we can combine rules to cut down on the number of functions.
        \item Some rules can be written iteratively rather than recursively. This is more effecient.
        \item Some rules do require a bit of care.
        \item Let's write the parser for the toyc language so you can see what I mean!
    \end{itemize}
\end{frame}

\end{document}
