% Uncomment for handout
\def\HANDOUT{}


\ifdefined\HANDOUT
\documentclass[handout]{beamer}
\usepackage{pgfpages}
\pgfpagesuselayout{4 on 1}[letterpaper,landscape,border shrink=5mm]
\else
\documentclass{beamer}
\fi

\mode<presentation>
{
  \usetheme{Warsaw}
  \definecolor{sered}{rgb}{0.78, 0.06, 0.18}
  \definecolor{richblack}{rgb}{0.0, 0.0, 0.0}
  \setbeamercolor{structure}{fg=sered,bg=richblack}
  %\setbeamercovered{transparent}
}


\usepackage[english]{babel}
\usepackage[latin1]{inputenc}
\usepackage{times}
\usepackage[T1]{fontenc}
\usepackage{tikz}
\usepackage{graphicx}
\usepackage[export]{adjustbox}
\usepackage{fancyvrb}
\usepackage{amsmath}
\usepackage{amssymb}
\usepackage{esvect}

\makeatletter
\newcommand{\imagesource}[1]{{\centering\hfill\break\hbox{\scriptsize Image Source:\thinspace{\tiny\itshape #1}}\par}}
\newcommand{\image}[3][\@nil]{%
        \def\tmp{#1}%
        \begin{center}
        \ifx\tmp\@nnil
            \includegraphics[max height = 0.55\textheight, max width = \textwidth]{images/#2}
        \else
            \includegraphics[max height = 0.50\textheight, max width = \textwidth]{images/#2}
            \linebreak
            #1
        \fi
        \linebreak
        {\tiny Image Source:\thinspace{\tiny #3}}
        \end{center}
}

\newenvironment{code}{%
 \VerbatimEnvironment
 \begin{adjustbox}{max width=\textwidth, max height=0.7\textheight}
 \begin{BVerbatim}
  }{
  \end{BVerbatim}
 \end{adjustbox}
}


\newenvironment{scaled}{%
 \begin{adjustbox}{max width=\textwidth, max height=0.7\textheight}
 }{
 \end{adjustbox}
}

\title{Constructing a Recursive Descent Parser}


\author{Robert Lowe}

\institute[Southeast Missouri State University] % (optional, but mostly needed)
{
  Department of Computer Science\\
  Southeast Missouri State University
}

\date[]{}
\subject{}

\pgfdeclareimage[height=1.0cm]{university-logo}{images/semo-logo}
\logo{\pgfuseimage{university-logo}}



\AtBeginSection[]
{
  \begin{frame}<beamer>{Outline}
    \tableofcontents[currentsection]
  \end{frame}
}


\begin{document}

\begin{frame}
  \titlepage
\end{frame}

\begin{frame}{Outline}
  \tableofcontents
\end{frame}


% Structuring a talk is a difficult task and the following structure
% may not be suitable. Here are some rules that apply for this
% solution: 

% - Exactly two or three sections (other than the summary).
% - At *most* three subsections per section.
% - Talk about 30s to 2min per frame. So there should be between about
%   15 and 30 frames, all told.

% - A conference audience is likely to know very little of what you
%   are going to talk about. So *simplify*!
% - In a 20min talk, getting the main ideas across is hard
%   enough. Leave out details, even if it means being less precise than
%   you think necessary.
% - If you omit details that are vital to the proof/implementation,
%   just say so once. Everybody will be happy with that.
\section{Preparing the Grammar}
\begin{frame}{Basic Steps}
    \begin{enumerate}
        \item Remove Left Recursion
        \item Compute First and Follow Sets
        \item Verify that rules are in LL(1) and fix any violations.
    \end{enumerate}
\end{frame}

\begin{frame}[fragile]{The Calc Parser Grammar}
\begin{code}
< Program >     ::= < Program > < Statement > 
                    | < Statement >

< Statement >   ::= < Expression > NEWLINE

< Expression >  ::= < Expression > PLUS < Term  >
                    | < Expression > MINUS < Term >
                    | < Term >

< Term >        ::= < Term  > TIMES  < Factor >
                    | < Term > DIVIDE < Factor >
                    | < Factor >

< Factor >      ::= < Base > POW < Factor >
                    | < Base >

< Base >        ::= LPAREN < Expression > RPAREN
                    | MINUS < Expression > 
                    | < Number >

< Number >      ::= INTLIT
                    | REALLIT
\end{code}
\end{frame}


\begin{frame}[fragile]{Non-Terminal -- \texttt{< Program >}}
\begin{block}{Original}
    \begin{code}
< Program >     ::= < Program > < Statement > 
                    | < Statement >
    \end{code}
\end{block}
\begin{itemize}
    \item<2-> This rule is left recursive.
    \item<3-> Technically an LL(1) violation.
    \item<4-> However, this is easily parsed as a while loop.
\end{itemize}
\end{frame}


\begin{frame}[fragile]{Non-Terminal -- \texttt{< Statement >}}
\begin{block}{Original}
    \begin{code}
< Statement >   ::= < Expression > NEWLINE
    \end{code}
\end{block}
\begin{itemize}
    \item<2-> This non-terminal is not left-recursive.
    \item<3-> There is only one rule, so there are no decisions to be made.
    \item<4-> There are no LL(1) violations in this non-terminal.
\end{itemize}
\end{frame}


\begin{frame}[fragile]{Non-Terminal -- \texttt{< Expression >}}
\begin{block}{Original}
    \begin{code}
< Expression >  ::= < Expression > PLUS < Term  >
                    | < Expression > MINUS < Term >
                    | < Term >
    \end{code}
\end{block}

\pause
\begin{block}{Left Recursion Eliminated}
    \begin{code}
< Expression >  ::= < Term > < Expression' >

< Expression' > ::= PLUS < Term  > < Expression' >
                    | MINUS < Term > < Expression' >
                    | ""
    \end{code}
\end{block}
\end{frame}


\begin{frame}[fragile]{Non-Terminal -- \texttt{< Term >}}
\begin{block}{Original}
    \begin{code}
< Term >        ::= < Term  > TIMES  < Factor >
                    | < Term > DIVIDE < Factor >
                    | < Factor >
    \end{code}
\end{block}

\pause
\begin{block}{Left Recursion Eliminated}
    \begin{code}
< Term >        ::= < Factor > < Term' >

< Term' >       ::= TIMES  < Factor > < Term' >
                    | DIVIDE < Factor > < Term' >
                    | ""
    \end{code}
\end{block}
\end{frame}


\begin{frame}[fragile]{Non-Terminal -- \texttt{< Factor >}}
\begin{block}{Original}
    \begin{code}
< Factor >      ::= < Base > POW < Factor >
                    | < Base >
    \end{code}
\end{block}
\begin{itemize}
    \item<2-> This non-terminal is not left-recursive.
    \item<3-> Technically, there is a violation as both rules begin with base.
    \item<4-> This is another situation which is easily fixed in code, so we will leave it.
\end{itemize}
\end{frame}


\begin{frame}[fragile]{Non-Terminal -- \texttt{< Base >}}
\begin{block}{Original}
    \begin{code}
< Base >        ::= LPAREN < Expression > RPAREN
                    | MINUS < Expression > 
                    | < Number >
    \end{code}
\end{block}
\begin{itemize}
    \item<2-> This non-terminal is not left-recursive.
    \item<3-> The first of each rule is distinct.
\end{itemize}
\end{frame}


\begin{frame}[fragile]{Non-Terminal -- \texttt{< Number >}}
\begin{block}{Original}
    \begin{code}
< Number >      ::= INTLIT
                    | REALLIT
    \end{code}
\end{block}
\begin{itemize}
    \item<2-> This non-terminal is not left-recursive.
    \item<3-> The first of each rule is distinct.
\end{itemize}
\end{frame}

\begin{frame}[fragile]{The RDP-Ready Representation of Calc}
\begin{columns}
\column{0.5\textwidth}
\begin{code}
< Program >     ::= < Program > < Statement > 
                    | < Statement >

< Statement >   ::= < Expression > NEWLINE

< Expression >  ::= < Term > < Expression' >

< Expression' > ::= PLUS < Term  > < Expression' >
                    | MINUS < Term > < Expression' >
                    | ""
                    
< Term >        ::= < Factor > < Term' >

< Term' >       ::= TIMES  < Factor > < Term' >
                    | DIVIDE < Factor > < Term' >
                    | ""

< Factor >      ::= < Base > POW < Factor >
                    | < Base >

< Base >        ::= LPAREN < Expression > RPAREN
                    | MINUS < Expression > 
                    | < Number >

< Number >      ::= INTLIT
                    | REALLIT
\end{code}

\column{0.5\textwidth}
\pause 
\textbf{Final Informal Evaluation}
\pause
\begin{enumerate}[<+->]
   \item Can we easily distinguish each rule using only simple if/while statements?
   \item Can we tell where each non-terminal ends using simple decisions?
   \item Can we make all decisions by only looking at the current token?
\end{enumerate}
\end{columns}
\end{frame}


\section{Constructing the Basic Parser}
\begin{frame}{The Parser Object}
    \begin{itemize}
        \item Public Interface
        \begin{itemize}
            \item A constructor which accepts a \texttt{Lexer} object.
            \item A function called \textt{parse} which throws an exception on parse errors.
        \end{itemize}
        \item Internal State
        \begin{itemize}
            \item \texttt{\_lexer} -- The token stream
            \item \texttt{\_curtok} -- The current token 
        \end{itemize}
    \end{itemize}
\end{frame}


\begin{frame}{Internal Functions}
    \begin{itemize}
        \item \texttt{next} -- Advance the lexer to the next token
        \item \texttt{has} -- Returns true if the current token matches the given token.
        \item \texttt{must\_be} -- Attempts to match a token and throws a \texttt{ParseError} exception on failure.
        \item \texttt{parse\_<non-terminal>} -- Mutually recursive functions derived from the BNF.
    \end{itemize}
\end{frame}


\begin{frame}[fragile]{The Public Interface}
\begin{code}
class Parser
{
public:
    Parser(Lexer &_lexer);
    ParseTree *parse();

protected:
    ...
private:
    ...
};
\end{code}
\end{frame}


\begin{frame}[fragile]{The Internal State}
\begin{code}
class Parser
{
public:
    ...
protected:
    ...
private:
    Lexer &_lexer;
    LexerToken _curtok;
};
\end{code}
\end{frame}


\begin{frame}[fragile]{The Protected Interface}
\begin{code}
class Parser
{
public:
    ...
protected:
    //token matches
    bool has(Token tok);
    void must_be(Token tok);

    //advanc the lexer
    void next();

    // non-terminal parse functions
    ParseTree *parse_program();
    ParseTree *parse_statement();
    ParseTree *parse_expression();
    ParseTree *parse_expression_prime();
    ParseTree *parse_term();
    ParseTree *parse_term_prime();
    ParseTree *parse_factor();
    ParseTree *parse_base();
    ParseTree *parse_number();

private:
    ...
};
\end{code}
\end{frame}

\begin{frame}[fragile]{Implementation of the Public Interface}
\begin{code}
// initalize the lexer and get the first token
Parser::Parser(Lexer &_lexer) : _lexer(_lexer)
{
    // Load up the lexer's token buffer.
    next();
}


// parse the program
ParseTree *Parser::parse()
{
    return parse_program();
}
\end{code}
\end{frame}


\begin{frame}[fragile]{Implementation of the Matching Functions}
\begin{code}
//token matches
bool Parser::has(Token tok)
{
    return _curtok == tok;
}


void Parser::must_be(Token tok)
{
    // Throw an exception if we don't match.
    if(not has(tok)) {
        // throw a parse error
        throw ParseError{_curtok};
    }
}
\end{code}
\end{frame}


\begin{frame}[fragile]{Implementation of \texttt{next}}
\begin{code}
//advance the lexer
void Parser::next()
{
    _curtok = _lexer.next();
}
\end{code}
\end{frame}


\section{Implementing Non-Terminal Rules}

\begin{frame}{How to Write the \texttt{parse\_<non-terminal>} Functions}
\begin{itemize}
    \item For any non-terminal, call the appropriate \texttt{parse\_...} function.
    \item For any terminal:
    \begin{itemize}
        \item If the terminal is one of several possible terminals, match with \texttt{has}.
        \item For any terminal which is the \textbf{only} remaining possibility, use the \texttt{must\_be} assertion.
        \item All terminal matches should be followed by a call to \texttt{next}.
    \end{itemize}
    \item Remember! You should be able to do everything with non-nested \texttt{if} and \texttt{while} logic.
    \item If any rule requires a lot of thought, you probably need to work on your grammar some more.
\end{itemize}
\end{frame}


\begin{frame}[fragile]{Implementation of \texttt{parse\_program}}
\begin{code}
/*
 * < Program >     ::= < Program > < Statement > 
 *                      | < Statement >
 */
ParseTree *Parser::parse_program()
{
    // Technically, this is not LL(1), but it is easy enough to handle 
    // this with a while loop
    while(not has(TEOF)) {
        parse_statement();
    }
    return nullptr;
}
\end{code}
\end{frame}


\begin{frame}[fragile]{Implementation of \texttt{parse\_statement}}
\begin{code}
/*
 * < Statement >   ::= < Expression > NEWLINE
 */
ParseTree *Parser::parse_statement()
{
    parse_expression();
    must_be(NEWLINE);
    next();

    return nullptr;
}
\end{code}
\end{frame}


\begin{frame}[fragile]{Implementation of \texttt{parse\_expression}}
\begin{code}
/*
 * < Expression >  ::= < Term > < Expression' >
 */
ParseTree *Parser::parse_expression()
{
    parse_term();
    parse_expression_prime();
    return nullptr;
}
\end{code}
\end{frame}


\begin{frame}[fragile]{Implementation of \texttt{parse\_expression\_prime}}
\begin{code}
/*
 * < Expression' > ::= PLUS < Term  > < Expression' >
 *                     | MINUS < Term > < Expression' >
 *                     | ""
 */
ParseTree *Parser::parse_expression_prime()
{
    if(has(PLUS)) {
        next();
        parse_term();
        parse_expression_prime();
    } else if(has(MINUS)) {
        next();
        parse_term();
        parse_expression_prime();
    }

    // nothing to do for the empty case
    return nullptr;
}
\end{code}
\end{frame}


\begin{frame}[fragile]{Implementation of \texttt{parse\_term}}
\begin{code}
/*
 * < Term >        ::= < Factor > < Term' >
 */
ParseTree *Parser::parse_term()
{
    parse_factor();
    parse_term_prime();

    return nullptr;
}
\end{code}
\end{frame}


\begin{frame}[fragile]{Implementation of \texttt{parse\_term\_prime}}
\begin{code}
/*
 * < Term' >       ::= TIMES  < Factor > < Term' >
 *                     | DIVIDE < Factor > < Term' >
 *                     | ""
 */
ParseTree *Parser::parse_term_prime()
{
    if(has(TIMES)) {
        next();
        parse_factor();
        parse_term_prime();
    } else if(has(DIVIDE)) {
        next();
        parse_factor();
        parse_term_prime();
    }

    //empty string, nothing to do
    return nullptr;
}
\end{code}
\end{frame}


\begin{frame}[fragile]{Implementation of \texttt{parse\_factor}}
\begin{code}
/*
 * < Factor >      ::= < Base > POW < Factor >
 *                     | < Base >
 */
ParseTree *Parser::parse_factor()
{
    parse_base();
    if(has(POW)) {
        next();
        parse_factor();
    }
    return nullptr;
}
\end{code}
\end{frame}


\begin{frame}[fragile]{Implementation of \texttt{parse\_factor}}
\begin{code}
/*
 * < Factor >      ::= < Base > POW < Factor >
 *                     | < Base >
 */
ParseTree *Parser::parse_factor()
{
    parse_base();
    if(has(POW)) {
        next();
        parse_factor();
    }
    return nullptr;
}
\end{code}
\end{frame}


\begin{frame}[fragile]{Implementation of \texttt{parse\_base}}
\begin{code}
/*
 * < Base >        ::= LPAREN < Expression > RPAREN
 *                     | MINUS < Expression > 
 *                     | < Number >
 */
ParseTree *Parser::parse_base()
{
    if(has(LPAREN)) {
        next();
        parse_expression();
        must_be(RPAREN);
        next();
    } else if(has(MINUS)) {
        next();
        parse_expression();
    } else {
        parse_number();
    }
    
    return nullptr;
}
\end{code}
\end{frame}


\begin{frame}[fragile]{Implementation of \texttt{parse\_number}}
\begin{code}
/*
 * < Number >      ::= INTLIT
 *                     | REALLIT
 */
ParseTree *Parser::parse_number()
{
    if(has(INTLIT)) {
        next();
        return nullptr;
    } else {
        must_be(REALLIT);
        next();
    }

    return nullptr;
}
\end{code}
\end{frame}


\begin{frame}{Suggested Exercises}
\begin{enumerate}
    \item For each non-terminal function, write pseudocode for the sort of parse tree it should produce.
    
    \item Write more example calc code to trigger each of the possible parse error conditions.

    \item Think about how we could improve error reporting. Are there any situations where we could give a more meaningful parse error?

    \item Further reduce the rules in the parser grammar to formally comply with LL(1). What are the advantages/disadvantages of implementing the fully compliant grammar?
\end{enumerate}
\end{frame}

\end{document}
