% Uncomment for handout
%\def\HANDOUT{}
%\def\PYTHON{}
\def\CPP{}


\ifdefined\HANDOUT
\documentclass[handout]{beamer}
\usepackage{pgfpages}
\pgfpagesuselayout{4 on 1}[letterpaper,landscape,border shrink=5mm]
\else
\documentclass{beamer}
\fi

\mode<presentation>
{
  \usetheme{Warsaw}
  \definecolor{sered}{rgb}{0.78, 0.06, 0.18}
  \definecolor{richblack}{rgb}{0.0, 0.0, 0.0}
  \setbeamercolor{structure}{fg=sered,bg=richblack}
  %\setbeamercovered{transparent}
}


\usepackage[english]{babel}
\usepackage[latin1]{inputenc}
\usepackage{times}
\usepackage[T1]{fontenc}
\usepackage{tikz}
\usepackage{graphicx}
\usepackage[export]{adjustbox}
\usepackage{fancyvrb}
\usepackage{amsmath}
\usepackage{amssymb}
\usepackage{esvect}

\makeatletter
\newcommand{\imagesource}[1]{{\centering\hfill\break\hbox{\scriptsize Image Source:\thinspace{\tiny\itshape #1}}\par}}
\newcommand{\image}[3][\@nil]{%
        \def\tmp{#1}%
        \begin{center}
        \ifx\tmp\@nnil
            \includegraphics[max height = 0.55\textheight, max width = \textwidth]{images/#2}
        \else
            \includegraphics[max height = 0.50\textheight, max width = \textwidth]{images/#2}
            \linebreak
            #1
        \fi
        \linebreak
        {\tiny Image Source:\thinspace{\tiny #3}}
        \end{center}
}

\newenvironment{code}{%
 \VerbatimEnvironment
 \begin{adjustbox}{max width=\textwidth, max height=0.7\textheight}
 \begin{BVerbatim}
  }{
  \end{BVerbatim}
 \end{adjustbox}
}


\newenvironment{scaled}{%
 \begin{adjustbox}{max width=\textwidth, max height=0.7\textheight}
 }{
 \end{adjustbox}
}

\title{Parsing}


\author{Robert Lowe}

\institute[Southeast Missouri State University] % (optional, but mostly needed)
{
  Department of Computer Science\\
  Southeast Missouri State University
}

\date[]{}
\subject{}

\pgfdeclareimage[height=1.0cm]{university-logo}{images/semo-logo}
\logo{\pgfuseimage{university-logo}}



\AtBeginSection[]
{
  \begin{frame}<beamer>{Outline}
    \tableofcontents[currentsection]
  \end{frame}
}


\begin{document}

\begin{frame}
  \titlepage
\end{frame}

\begin{frame}{Outline}
  \tableofcontents
\end{frame}


% Structuring a talk is a difficult task and the following structure
% may not be suitable. Here are some rules that apply for this
% solution: 

% - Exactly two or three sections (other than the summary).
% - At *most* three subsections per section.
% - Talk about 30s to 2min per frame. So there should be between about
%   15 and 30 frames, all told.

% - A conference audience is likely to know very little of what you
%   are going to talk about. So *simplify*!
% - In a 20min talk, getting the main ideas across is hard
%   enough. Leave out details, even if it means being less precise than
%   you think necessary.
% - If you omit details that are vital to the proof/implementation,
%   just say so once. Everybody will be happy with that.
\section{The Parsing Problem}
\begin{frame}{What is Parsing?}
    \begin{itemize}
        \item Parsing is working the generating problem backwards.
        \item Given a sentence in a language, produce a parse tree for that sentence.
        \item If no parse tree can be found, we reject the sentence.
    \end{itemize}
\end{frame}

\begin{frame}{Basic Approaches to Parsing}
    \begin{description}
        \item[Top-Down] We start with the root of the tree and build toward the leaves.
        \begin{itemize}
            \item Easiest to Implement
            \item Generally Requires Grammar Constraints
        \end{itemize}
        \item[Bottom-Up] We start at the leaves and build toward the root.
        \begin{itemize}
            \item Difficult to Implement, Usually Machine Generated
            \item Has Fewer Grammar Constraints
        \end{itemize}
    \end{description}
\end{frame}

\begin{frame}{A Top-Down Example: \texttt{2+3}}
    \begin{columns}
    \column{0.5\textwidth}
    \begin{enumerate}[<+->]
        \item expr
        \item expr \texttt{+} term
        \item term
        \item num
        \item \texttt{2}
        \item num
        \item \texttt{3}
    \end{enumerate}
    \column{0.5\textwidth}
    \begin{center}
        \begin{scaled}
            \begin{tikzpicture}
                    \node{exp}
                        child {node {expr}
                            child {node {term}
                                child {node {num}
                                    child {node {\texttt{2}}}}
                            }
                        }
                        child {node {\texttt{+}}}
                        child {node {term}
                            child {node {num}
                                child {node {\texttt{3}}}
                            }
                        }
            \end{tikzpicture}
        \end{scaled}
    \end{center}
    \end{columns}
\end{frame}

\begin{frame}{A Bottom-Up Example: \texttt{2+3}} 
    \begin{columns}
    \column{0.5\textwidth}
    \begin{enumerate}[<+->]
        \item \texttt{3}
        \item num
        \item term 
        \item \textt{2}
        \item num
        \item term
        \item expr
        \item expr \texttt{+} term
        \item exp
    \end{enumerate}
    \column{0.5\textwidth}
    \begin{center}
        \begin{scaled}
            \begin{tikzpicture}
                    \node{exp}
                        child {node {expr}
                            child {node {term}
                                child {node {num}
                                    child {node {\texttt{2}}}}
                            }
                        }
                        child {node {\texttt{+}}}
                        child {node {term}
                            child {node {num}
                                child {node {\texttt{3}}}
                            }
                        }
            \end{tikzpicture}
        \end{scaled}
    \end{center}
    \end{columns}
\end{frame}


\section{Parsing Algorithms}
\begin{frame}{The Complexity of Parsing}
    \begin{itemize}
        \item General parsing has complexity $O(n^3)$.
        \item General parsing also has a $O(n^3)$ backtracking requirement.
        \item With a few constraints, we can parse in $O(n)$ time!
        \item Most computer languages can be parsed in linear time.
    \end{itemize}
\end{frame}

\begin{frame}{Classifications of Parsers}  
    Parsers are classified according to the {\bf direction of scanning}, and the {\bf direction of derivation}.
    \begin{description}
        \item[LL] - Left-to-Right Scanning, Left most derivation.
        \item[LR] - Left-to-Right Scanning, Right most derivation.
    \end{description}
    Parsers are further classified by how many tokens they must {\bf look ahead}.
    \begin{description}
        \item[LL$(k)$] - LL Parser with constant $k$ tokens of look-ahead.
        \item[LR$(k)$] - LR Parser with constant $k$ tokens of look-ahead.
        \item[LALR] - A simplified LR$(k)$ parser.
    \end{description}
\end{frame}
\begin{frame}[fragile]{Recursive Descent Parsing}
    \begin{itemize}
        \item Recursive Descent Parsers are derived from the BNF.
        \item Each non-terminal is expressed as a mutually recursive function.
        \item RD Parsers have no backtracking.
        \item Only apply to $LL(1)$ grammars!
    \end{itemize}
\end{frame}

\begin{frame}{LALR Parsers}
    \begin{itemize}
        \item Most typical example is the LALR$(1)$ parser.
        \item Use a state table to maintain a list of possible derivations.
        \item Performs a {\bf shift-reduce} operation in order to decrease possiblities list.
        \item These are generally machine generated.
    \end{itemize}
\end{frame}

\begin{frame}{Compiler Compilers --- Parser Generators}
    \begin{itemize}
        \item Compiler Compilers generate parsers from a (typically) BNF-like syntax.
        \item Most commercial compilers are created using these tools.
        \item Notable Examples:
        \begin{itemize}
            \item YACC - Yet Another Compiler Compiler (C Parser Generator)
            \item JACC - Just Another Compiler Compiler (Java Parser Generator)
            \item PLY - Python-Lex-Yacc (Python Lexer and Parser Generator)
        \end{itemize}
    \end{itemize}
\end{frame}

\begin{frame}[fragile]{YACC Example}
\begin{code}
%left '+' '-'
%left '*' '/' 

%% 
expr:    '(' expr ')'    { $$ = $2; }
         | expr '*' expr { $$.a = $1.a * $3.a; }
         | expr '/' expr { $$.a = $1.a / $3.a; }
         | expr '+' expr { $$.a = $1.a + $3.a; }
         | expr '-' expr { $$.a = $1.a - $3.a; }
\end{code}
\end{frame}

\section{General Recursive Descent Parsing}

\begin{frame}{A Sample -- As an Expression}
Expression: \texttt{3*(2-5)}
\begin{columns}
    \column{0.7\textwidth}
     
    \centering
    \begin{adjustbox}{max width=\textwidth, totalheight=0.8\textheight}
        \begin{tikzpicture}
            \node {$E$}
                child { node {$T$}
                    child { node {$T$}
                        child{ node {$F$} 
                            child { node {\textttt{3}}}
                        }
                    }    
                    child { node {\textttt{*}}}    
                    child { node {$F$}
                        child { node {\texttt{(}}}
                        child { node {$E$} 
                            child {node { $E$ }
                                child {node {$T$}
                                    child {node {$F$}
                                        child {node {\texttt{2}}}
                                    }
                                }
                            }
                            child {node {\texttt{-}}}
                            child {node {$T$}
                                child {node {$F$}
                                    child {node {\textttt{5}}}
                                }
                            }
                        }
                        child { node {\texttt{)}}}
                    }
                }
        \end{tikzpicture}
    \end{adjustbox}
    
    \column{0.3\textwidth}
    \begin{enumerate}
        \item $E \rightarrow E+T$
        \item $E \rightarrow E-T$
        \item $E \rightarrow T$
        \item $T \rightarrow T*F$
        \item $T \rightarrow T/F$
        \item $T \rightarrow F$
        \item $F \rightarrow (E)$
        \item $F \rightarrow Number$
    \end{enumerate}    
\end{columns}
\end{frame}


\begin{frame}{A Sample -- As a Token Stream}
\begin{adjustbox}{width=\textwidth}
\texttt{NUMBER TIMES LPAREN NUMBER MINUS NUMBER RPAREN}
\end{adjustbox}

\begin{columns}
    \column{0.7\textwidth}
     
    \centering
    \begin{adjustbox}{max width=\textwidth, totalheight=0.8\textheight}
        \begin{tikzpicture}
            \node {$E$}
                child { node {$T$}
                    child { node {$T$}
                        child{ node {$F$} 
                            child { node {\textttt{NUMBER}}}
                        }
                    }    
                    child { node {\textttt{TIMES}}}    
                    child { node {$F$}
                        child { node {\texttt{LPAREN}}}
                        child { node {$E$} 
                            child {node { $E$ }
                                child {node {$T$}
                                    child {node {$F$}
                                        child {node {\texttt{NUMBER}}}
                                    }
                                }
                            }
                            child {node {\texttt{MINUS}}}
                            child {node {$T$}
                                child {node {$F$}
                                    child {node {\textttt{NUMBER}}}
                                }
                            }
                        }
                        child { node {\texttt{RPAREN}}}
                    }
                }
        \end{tikzpicture}
    \end{adjustbox}
    
    \column{0.3\textwidth}
    \begin{enumerate}
        \item $E \rightarrow E+T$
        \item $E \rightarrow E-T$
        \item $E \rightarrow T$
        \item $T \rightarrow T*F$
        \item $T \rightarrow T/F$
        \item $T \rightarrow F$
        \item $F \rightarrow (E)$
        \item $F \rightarrow Number$
    \end{enumerate}    
\end{columns}
\end{frame}

\begin{frame}{How can we generate the tree?}
\begin{adjustbox}{width=\textwidth}
\texttt{NUMBER TIMES LPAREN NUMBER MINUS NUMBER RPAREN}
\end{adjustbox}

\begin{columns}
    \column{0.7\textwidth}
    A simple approach:
    \begin{itemize}
        \item Try each rule in turn.
        \item Expand until a match becomes impossible.
        \item Backtrack and repeat until all derivations are expanded.
    \end{itemize}
    
    \column{0.3\textwidth}
    \begin{enumerate}
        \item $E \rightarrow E+T$
        \item $E \rightarrow E-T$
        \item $E \rightarrow T$
        \item $T \rightarrow T*F$
        \item $T \rightarrow T/F$
        \item $T \rightarrow F$
        \item $F \rightarrow (E)$
        \item $F \rightarrow Number$
    \end{enumerate}    
\end{columns}
\end{frame}

\begin{frame}{We need determinism!}
    \begin{itemize}
        \item We need to make definitive choices, backtracking is expensive!
        \item We have the following information to go on:
        \begin{itemize}
            \item A buffer of tokens.
            \item Where we are in the parsing process.
        \end{itemize}
        \item The goal is to do this scanning only 1 token at a time.
        \item We need an LL(1) grammar to do this!
    \end{itemize}
\end{frame}


\section{LL(1) Grammars}
\begin{frame}{LL(1) Grammars}
    \begin{block}{Definition}
        An {\bf LL(1) Grammar} is a grammar where rules can be correctly selected using
        only 1 token of look-ahead.
    \end{block}
    
    \begin{itemize}
        \item Not all languages can have an LL(1) grammar.
        \item Properties:
        \begin{itemize}
            \item Non-Ambiguous
            \item Not Left Recursive
        \end{itemize}
    \end{itemize}
\end{frame}

\begin{frame}{$First$ and $Follow$}
    \begin{description}
        \item[$First(A)$] The set of all terminals which can appear at the beginning of some derivation of $A$.
        \item[$Follow(A)$] The set of all terminals which can appear to the right of a derivation of $A$.
    \end{description}
\end{frame}

\begin{frame}{Formal Verification of LL(1)}
\[
G = \langle \Sigma,N,S,P\rangle
\]

A context free grammar, $G$ is an LL(1) grammar if and only if
\begin{itemize}
    \item For every $A\in N$ and strings $\alpha,\beta$ such that $\alpha \neq \beta$ and $A\rightarrow \alpha|\beta \in P$:
    \begin{itemize}
        \item $First(\alpha) \cap First(\beta)=\emptyset$
        \item If $\alpha \Rightarrow \epsilon$ then $First(\beta)\cap Follow(A) = \emptyset$
    \end{itemize}
\end{itemize}
\end{frame}

\begin{frame}{What about our Expression Grammar?}
\begin{columns}
    \column{0.7\textwidth}
    \begin{itemize}[<+->]
        \item $First(E) = \{Number, (\}$
        \item $First(T) = \{Number, (\}$
        \item $First(F) = \{Number,(\}$
        \item $First(E+T) = First(E)$
        \item $First(E-T) = First(E)$
        \item $First(T*F) = First(T)$
        \item $First(T/F) = First(T)$
        \item So this grammar is not LL(1)!
    \end{itemize}
    
    \column{0.3\textwidth}
    \begin{enumerate}
        \item $E \rightarrow E+T$
        \item $E \rightarrow E-T$
        \item $E \rightarrow T$
        \item $T \rightarrow T*F$
        \item $T \rightarrow T/F$
        \item $T \rightarrow F$
        \item $F \rightarrow (E)$
        \item $F \rightarrow Number$
    \end{enumerate}    
\end{columns}
\end{frame}

\begin{frame}{Eliminating Left Recursion}
\begin{columns}
    \column{0.5\textwidth}
    \begin{block}{Left Recursive Rule}
        \[
        A \rightarrow A\alpha_1 | A\alpha_2 | \ldots | A\alpha_n | \beta
        \]
    \end{block}
    \column{0.5\textwidth}
    \begin{block}{Non-Left Recursive Rule}
        \begin{align*}
            A &\rightarrow \beta A'\\
            A' & \rightarrow \alpha_1 A' | \alpha_2 A' | \ldots | \alpha_n A'| \epsilon
        \end{align*}
    \end{block}
\end{columns}
\begin{itemize}
    \item Direct left-recursion can be removed by the above replacement.
    \item We may use substitution on transitive left recursive grammars (these are rare)
\end{itemize}
\end{frame}

\begin{frame}{Eliminating Left Recursion from E}
    \begin{enumerate}[<+->]
        \item $E \rightarrow E+T | E-T | T$
        \item $\alpha_1 = +T, \alpha_2=-T, \beta=T$
        \item Perform the Replacement:
        \begin{itemize}
            \item $E \rightarrow TE'$
            \item $E' \rightarrow +TE' | -TE' | \epsilon$
        \end{itemize}
    \end{enumerate}
\end{frame}

\begin{frame}{The LL(1) Expression Grammar}
\begin{columns}
\column{0.7\textwidth}
    \begin{itemize}[<+->]
        \item $A \rightar \alpha | \beta$ s.t. $\alpha \neq \beta$ means that $First(\alpha)\cap First(\beta) = \emptyset$
        \begin{itemize}
            \item $First(TE') = First(T)$
            \item $First(+TE') = \{+\}$
            \item $First(-TE') = \{-\}$
            \item Also holds for $E'$, $T$, $T'$, $F$
        \end{itemize}
        \item $\alpha \Rightarrow \epsilon$ means that $First(\beta)\cap Follow(A)=\emptyset$
        \begin{itemize}
            \item $Follow(E')=Follow(E)$
            \item $Follow(E)=\{)\}$
            \item $Follow(T')=Follow(T)$
            \item $Follow(T)=\{+,-\}$
        \end{itemize}
        \item This grammar is therefore an LL(1) Grammar
    \end{itemize}
\column{0.3\textwidth}
\begin{enumerate}
    \item $E \rightarrow TE'$
    \item $E' \rightarrow +TE'$
    \item $E' \rightarrow -TE'$
    \item $E' \rightarrow \epsilon$
    \item $T \rightarrow FT'$
    \item $T' \rightarrow *FT'$
    \item $T' \rightarrow /FT'$
    \item $T' \rightarrow \epsilon$
    \item $F \rightarrow (E)$
    \item $F \rightarrow Number$
\end{enumerate}
\end{columns}
\end{frame}

\ifdefined\PYTHON
\begin{frame}[fragile]{LL(1) Recursive Descent Parsing}
\begin{code}
def E(self):
  self.T()
  self.E_prime()

def E_prime(self):
    if self.lexer.cur_tok.token == Token.PLUS:
        self.lexer.next()
        self.T()
        self.T_prime()
    elif self.lexer.cur_tok.token == Token.MINUS:
        self.lexer.next()
        self.T()
        self.T_prime()
    # epsilon requires no code, we just don't recurse.
\end{code}
\end{frame}
\fi

\ifdefined\CPP
\begin{frame}[fragile]{LL(1) Recursive Descent Parsing}
\begin{code}
void Parser::parse_E() {
    parse_T();
    parse_E_prime();
}


void Parser::parse_E_prime() {
    if(lexer.current() == PLUS) {
        lexer.next();
        parse_T();
        parse_T_prime();
    } else if(lexer.current() == MINUS) {
        lexer.next();
        parse_T();
        parse_T_prime();
    }
}
\end{code}
\end{frame}
\fi

\begin{frame}{Suggested Exercises}
    \begin{enumerate}
        \item Eliminate left recursion from calc.
        \item Verify that the non-left recursive calc is LL(1).
        \item Sketch out the functions that will be needed to parse calc.
    \end{enumerate} 
\end{frame}


\end{document}

