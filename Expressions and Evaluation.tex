% Uncomment for handout
\def\HANDOUT{}


\ifdefined\HANDOUT
\documentclass[handout]{beamer}
\usepackage{pgfpages}
\pgfpagesuselayout{4 on 1}[letterpaper,landscape,border shrink=5mm]
\else
\documentclass{beamer}
\fi

\mode<presentation>
{
  \usetheme{Warsaw}
  \definecolor{sered}{rgb}{0.78, 0.06, 0.18}
  \definecolor{richblack}{rgb}{0.0, 0.0, 0.0}
  \setbeamercolor{structure}{fg=sered,bg=richblack}
  %\setbeamercovered{transparent}
}


\usepackage[english]{babel}
\usepackage[latin1]{inputenc}
\usepackage{times}
\usepackage[T1]{fontenc}
\usepackage{tikz}
\usepackage{graphicx}
\usepackage[export]{adjustbox}
\usepackage{fancyvrb}
\usepackage{amsmath}
\usepackage{amssymb}
\usepackage{esvect}

\makeatletter
\newcommand{\imagesource}[1]{{\centering\hfill\break\hbox{\scriptsize Image Source:\thinspace{\tiny\itshape #1}}\par}}
\newcommand{\image}[3][\@nil]{%
        \def\tmp{#1}%
        \begin{center}
        \ifx\tmp\@nnil
            \includegraphics[max height = 0.55\textheight, max width = \textwidth]{images/#2}
        \else
            \includegraphics[max height = 0.50\textheight, max width = \textwidth]{images/#2}
            \linebreak
            #1
        \fi
        \linebreak
        {\tiny Image Source:\thinspace{\tiny #3}}
        \end{center}
}

\newenvironment{code}{%
 \VerbatimEnvironment
 \begin{adjustbox}{max width=\textwidth, max height=0.7\textheight}
 \begin{BVerbatim}
  }{
  \end{BVerbatim}
 \end{adjustbox}
}


\newenvironment{scaled}{%
 \begin{adjustbox}{max width=\textwidth, max height=0.7\textheight}
 }{
 \end{adjustbox}
}

\title{Expressions and Evaluation}


\author{Robert Lowe}

\institute[Southeast Missouri State University] % (optional, but mostly needed)
{
  Department of Computer Science\\
  Southeast Missouri State University
}

\date[]{}
\subject{}

\pgfdeclareimage[height=1.0cm]{university-logo}{images/semo-logo}
\logo{\pgfuseimage{university-logo}}



\AtBeginSection[]
{
  \begin{frame}<beamer>{Outline}
    \tableofcontents[currentsection]
  \end{frame}
}


\begin{document}

\begin{frame}
  \titlepage
\end{frame}

\begin{frame}{Outline}
  \tableofcontents
\end{frame}


% Structuring a talk is a difficult task and the following structure
% may not be suitable. Here are some rules that apply for this
% solution: 

% - Exactly two or three sections (other than the summary).
% - At *most* three subsections per section.
% - Talk about 30s to 2min per frame. So there should be between about
%   15 and 30 frames, all told.

% - A conference audience is likely to know very little of what you
%   are going to talk about. So *simplify*!
% - In a 20min talk, getting the main ideas across is hard
%   enough. Leave out details, even if it means being less precise than
%   you think necessary.
% - If you omit details that are vital to the proof/implementation,
%   just say so once. Everybody will be happy with that.

\section{Expression Effects}
\begin{frame}{Effects and Side Effects}
    \begin{block}{Definitions}
        \begin{itemize} 
            \item An \textbf{effect} of an expression is the result which replaces the expression.
            \item A \textbf{side effect} of an expression is a result that happens along with this replacement. This is usually some external state change or input/output result.
        \end{itemize}
    \end{block}
    \begin{itemize}
        \item Assignment Statements
        \begin{itemize}
            \item Effect: Returns the value that was assigned.
            \item Side Effect: Stores a value in memory.
        \end{itemize}
        \item Arithmetic Expressions
        \begin{itemize}
            \item Effect: The value of the arithmetic evaluation.
            \item Side Effect: None
        \end{itemize}
    \end{itemize}
\end{frame}

\begin{frame}[fragile]{Example: C++}
\begin{code}
Effect                                   Side Effect
======                                   ===========
cout << "Hello, World" << endl;      ==> Displays Hello, World
     ||
     \/
     cout << endl;                   ==> Displays End Of Line
          ||
          \/
         cout 
\end{code}
\end{frame}

\begin{frame}{Functional Programming and Expressions}
    \begin{itemize}
        \item In functional programming, expressions have no side-effects.
        \item This allows expressions to be processed independently, and sometimes out of order.
    \end{itemize}
\end{frame}

\section{Expression Types}

\begin{frame}{Arithmetic Expressions}
    \begin{itemize}
        \item Arithmetic expressions evaluate to their arithmetic value.
        \item Typically, these have no side effects.
    \end{itemize}
\end{frame}

\begin{frame}{Overloaded Operators}
    \begin{itemize}
        \item Operator overloading allows user types to participate in the language's expression syntax.
        \item This can be abused when programmers create unexpected side effects to operations.
        \item Generally, operator overloading is constrained in most modern languages.
    \end{itemize}
\end{frame}

\begin{frame}[fragile]{Assignment as an Expression}
\begin{code}
Effect             Side Effect
======             ===========
x = y = z = 2; ==> z is assigned 2
         ||
         \/
 x = y = 2;    ==> y is assigned 2
      ||
      \/
    x = 2;     ==> x is assigned 2
      ||
      \/
      2
\end{code}
\end{frame}

\begin{frame}[fragile]{Type Conversion}
\begin{itemize}
    \item Type conversions can be implicit, as in type coercion.
    \item Type conversions can also be explicit, as in C/C++/Java's typecast operators:
    \newline
    \begin{code}
    
Effect
======
(int) 9.0
    ||
    \/
    9
    
    \end{code}
    \item Typically type conversion have no side effects.
\end{itemize}
\end{frame}

\section{Evaluation Strategies}
\begin{frame}{Operand Evaluation}
    \begin{itemize}
        \item For left associative operators: process all operands left to right then evaluate the expression.
        \item For right associative operators: process all operands right to left, then evaluate the expression.
    \end{itemize}
\end{frame}

\begin{frame}{Short Circuit Evaluation}
    \begin{itemize}
        \item \textbf{Short circuit evaluation} is where we stop evaluating operands if the answer becomes inevitable.
        \item Typically done with boolean operations: \texttt{&&}, \texttt{||}
        \[
        \begin{array}{cc|c|c}
        a & b & a \&\& b & a || b \\
        \hline
        F & F & F & F \\
        F & T & F & T \\
        T & F & F & T \\
        T & T & T & T \\
        \end{array}
        \]
        \item Sometimes done with multiplication.
    \end{itemize}
\end{frame}

\begin{frame}[fragile]{Guard Expressions}
    \begin{itemize}
        \item Guard expressions are where use short circuits to guard the execution of an expression.
        \item Sometimes these are language features, but mostly these are just a consequence of short-circuit evaluation.
    \end{itemize}
    \begin{code}
    
    if i < len(ar) and ar[i] != 0:
      print("Not zero")
    \end{code}
\end{frame}

\begin{frame}{Reading Assignment}
    \begin{itemize}
        \item Read Chapter 7 -- Expressions and Assignment Statements 
    \end{itemize}
\end{frame}

\end{document}
