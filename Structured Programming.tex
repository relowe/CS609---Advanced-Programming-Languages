% Uncomment for handout
\def\HANDOUT{}


\ifdefined\HANDOUT 
\documentclass[handout]{beamer}
\usepackage{pgfpages}
\pgfpagesuselayout{4 on 1}[letterpaper,landscape,border shrink=5mm]
\else
\documentclass{beamer}
\fi

\mode<presentation>
{
  \usetheme{Warsaw}
  \definecolor{sered}{rgb}{0.78, 0.06, 0.18}
  \definecolor{richblack}{rgb}{0.0, 0.0, 0.0}
  \setbeamercolor{structure}{fg=sered,bg=richblack}
  %\setbeamercovered{transparent}
}


\usepackage[english]{babel}
\usepackage[latin1]{inputenc}
\usepackage{times}
\usepackage[T1]{fontenc}
\usepackage{tikz}
\usepackage{graphicx}
\usepackage[export]{adjustbox}
\usepackage{fancyvrb}
\usepackage{amsmath}
\usepackage{amssymb}
\usepackage{esvect}

\makeatletter
\newcommand{\imagesource}[1]{{\centering\hfill\break\hbox{\scriptsize Image Source:\thinspace{\tiny\itshape #1}}\par}}
\newcommand{\image}[3][\@nil]{%
        \def\tmp{#1}%
        \begin{center}
        \ifx\tmp\@nnil
            \includegraphics[max height = 0.55\textheight, max width = \textwidth]{images/#2}
        \else
            \includegraphics[max height = 0.50\textheight, max width = \textwidth]{images/#2}
            \linebreak
            #1
        \fi
        \linebreak
        {\tiny Image Source:\thinspace{\tiny #3}}
        \end{center}
}

\newenvironment{code}{%
 \VerbatimEnvironment
 \begin{adjustbox}{max width=\textwidth, max height=0.7\textheight}
 \begin{BVerbatim}
  }{
  \end{BVerbatim}
 \end{adjustbox}
}


\newenvironment{scaled}{%
 \begin{adjustbox}{max width=\textwidth, max height=0.7\textheight}
 }{
 \end{adjustbox}
}

\title{Structured Programming}


\author{Robert Lowe}

\institute[Southeast Missouri State University] % (optional, but mostly needed)
{
  Department of Computer Science\\
  Southeast Missouri State University
}

\date[]{}
\subject{}

\pgfdeclareimage[height=1.0cm]{university-logo}{images/semo-logo}
\logo{\pgfuseimage{university-logo}}



\AtBeginSection[]
{
  \begin{frame}<beamer>{Outline}
    \tableofcontents[currentsection]
  \end{frame}
}


\begin{document}

\begin{frame}
  \titlepage
\end{frame}

\begin{frame}{Outline}
  \tableofcontents
\end{frame}


% Structuring a talk is a difficult task and the following structure
% may not be suitable. Here are some rules that apply for this
% solution: 

% - Exactly two or three sections (other than the summary).
% - At *most* three subsections per section.
% - Talk about 30s to 2min per frame. So there should be between about
%   15 and 30 frames, all told.

% - A conference audience is likely to know very little of what you
%   are going to talk about. So *simplify*!
% - In a 20min talk, getting the main ideas across is hard
%   enough. Leave out details, even if it means being less precise than
%   you think necessary.
% - If you omit details that are vital to the proof/implementation,
%   just say so once. Everybody will be happy with that.

\section{Pure Imperative Programming}

\begin{frame}{The Imperative Paradigm}
\begin{block}{Imperative Programming}
Imperative programming is completely about state.
\end{block} 
\begin{itemize}
    \item Imperative Programs control state.
    \item Typically jumps to location in code to perform logic.
    \item Most machine code ISAs are imperative.
    \item The oldest form of programming.
\end{itemize}
\end{frame}


\begin{frame}[fragile]{BASIC}
\begin{columns}
\column{0.5\textwidth}
\begin{itemize}
    \item Beginner's All Purpose Symbolic Instruction Code (BASIC) is a high-level imperative language.
    \item There is one loop structure \texttt{FOR}.
    \item All other logic is controlled by \texttt{GOTO} and conditional jumps.
\end{itemize}
\column{0.5\textwidth}
\begin{code}
10  REM Fizz Buzz, The classic puzzle
20  INPUT "How far should I count? "; MAX
30  FOR I = 1 TO MAX
40  OUT$ = ""
50  IF I MOD 3 <> 0 THEN GOTO 70
60  OUT$ = OUT$ + "Fizz "
70  IF I MOD 5 <> 0 THEN GOTO 90
80  OUT$ = OUT$ + "Buzz"
90  IF OUT$ = "" THEN GOTO 120
100 PRINT OUT$
110 GOTO 130
120 PRINT I
130 NEXT I
\end{code}
\end{columns}    
\end{frame}


\begin{frame}[fragile]{The Problem}
\begin{columns}
\column{0.5\textwidth}
\begin{itemize}
    \item It is difficult to see the overall structure of an imperative program.
    \item Analyzing code means hand-tracing its path of execution.
    \item Maintainability and Readability suffer.
    \item The problem is often called ``Spaghetti Code.''
\end{itemize}
\column{0.5\textwidth}
\begin{code}
10  REM Fizz Buzz, The classic puzzle
20  INPUT "How far should I count? "; MAX
30  FOR I = 1 TO MAX
40  OUT$ = ""
50  IF I MOD 3 <> 0 THEN GOTO 70
60  OUT$ = OUT$ + "Fizz "
70  IF I MOD 5 <> 0 THEN GOTO 90
80  OUT$ = OUT$ + "Buzz"
90  IF OUT$ = "" THEN GOTO 120
100 PRINT OUT$
110 GOTO 130
120 PRINT I
130 NEXT I
\end{code}
\end{columns}    
    
\end{frame}

\section{Structured Programming}
\begin{frame}{Go To Considered Harmful}
\begin{columns}
\column{0.6\textwidth}
\begin{itemize}
    \item Edsger Dijkstra published an open letter to CACM in 1968.
    \item Advocates against the use of Go To statements as they obscure the intent of the programming.
    \item Coined the term ``structured programming''
    \item Most modern languages follow the structured programming paradigm.
\end{itemize}
\column{0.4\textwidth}
\image[Edsger Dijkstra\newline1930 -- 2002]{dijkstra}{Wikipedia}
\end{columns}
\end{frame}


\begin{frame}{Elements of Structured Programming}
    \begin{itemize}
        \item No explicit jumps.
        \item Branch Structures
        \item Loop Structures
        \item Blocks
        \item Subroutines and Functions
    \end{itemize}
\end{frame}


\begin{frame}[fragile,t]{Branches and Loops}
\begin{columns}[t]
\column{0.5\textwidth}
\begin{block}{Branches}
\begin{code}
if( condition ) {
  // Executed if True
} else {
  // Executed if False
}
\end{code}
\end{block}
\column{0.5\textwidth}
\begin{block}{Loops}
\begin{code}
while(condition) {
   //body
}

do {
   //body
} while(condition);

for(init; condition; update) {
  //body
}
\end{code}
\end{block}
\end{columns}
\end{frame}

\begin{frame}[fragile]{Blocks}
\begin{columns}
\column{0.5\textwidth}
\begin{itemize}
    \item Blocks form the bodies of loops, branches, and procedures.
    \item Blocks can be nested.
    \item Blocks define neat regions of scope and intention.
\end{itemize}
\column{0.5\textwidth}
\begin{block}{Blocks}
\begin{code}
for(int i=1; i<=10; i++) {
  System.out.println(i);
}
\end{code}
\end{block}
\end{columns}
\end{frame}


\begin{frame}[fragile]{Subroutines}
\begin{columns}
\column{0.5\textwidth}
\begin{itemize}
    \item Subroutines have exactly one entry point.
    \item Subroutines have exactly one exit point.
\end{itemize}
\column{0.5\textwidth}
\begin{block}{Subroutine}
\begin{code}
void count(int max) 
{
    for(int i=1; i<=max; i++) {
        System.out.println(i);
    }
}
\end{code}
\end{block}
\end{columns}
\end{frame}


\begin{frame}{Why \texttt{goto} Still Exists}
\begin{columns}
\column{0.6\textwidth}
\begin{itemize}
    \item There are some common operations which do not work well in pure structured languages.
    \item Error handling is one prominent example.
    \item We frequently violate structured programming by:
    \begin{itemize}
        \item Throwing exceptions.
        \item Having multiple exit points to subroutines.
        \item Using \texttt{goto} to have a common error handler.
    \end{itemize}
\end{itemize}
\column{0.4\textwidth}
\image[Donald Knuth\newline1938--]{knuth}{Wikipedia}
\end{columns}
\end{frame}

\begin{frame}{Reading Assignment}
    \begin{itemize}
        \item \textit{Go To Considered Harmful} by Edsger Dijkstra
        \item \textit{Structured Programming with Go To} by Donald Knuth
    \end{itemize}
\end{frame}

\end{document}
