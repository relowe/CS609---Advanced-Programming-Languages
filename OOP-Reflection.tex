% Uncomment for handout
\def\HANDOUT{}


\ifdefined\HANDOUT
\documentclass[handout]{beamer}
\usepackage{pgfpages}
\pgfpagesuselayout{4 on 1}[letterpaper,landscape,border shrink=5mm]
\else
\documentclass{beamer}
\fi

\mode<presentation>
{
  \usetheme{Warsaw}
  \definecolor{sered}{rgb}{0.78, 0.06, 0.18}
  \definecolor{richblack}{rgb}{0.0, 0.0, 0.0}
  \setbeamercolor{structure}{fg=sered,bg=richblack}
  %\setbeamercovered{transparent}
}


\usepackage[english]{babel}
\usepackage[latin1]{inputenc}
\usepackage{times}
\usepackage[T1]{fontenc}
\usepackage{tikz}
\usepackage{graphicx}
\usepackage[export]{adjustbox}
\usepackage{fancyvrb}
\usepackage{amsmath}
\usepackage{amssymb}
\usepackage{esvect}

\makeatletter
\newcommand{\imagesource}[1]{{\centering\hfill\break\hbox{\scriptsize Image Source:\thinspace{\tiny\itshape #1}}\par}}
\newcommand{\image}[3][\@nil]{%
        \def\tmp{#1}%
        \begin{center}
        \ifx\tmp\@nnil
            \includegraphics[max height = 0.55\textheight, max width = \textwidth]{images/#2}
        \else
            \includegraphics[max height = 0.50\textheight, max width = \textwidth]{images/#2}
            \linebreak
            #1
        \fi
        \linebreak
        {\tiny Image Source:\thinspace{\tiny #3}}
        \end{center}
}

\newenvironment{code}{%
 \VerbatimEnvironment
 \begin{adjustbox}{max width=\textwidth, max height=0.7\textheight}
 \begin{BVerbatim}
  }{
  \end{BVerbatim}
 \end{adjustbox}
}


\newenvironment{scaled}{%
 \begin{adjustbox}{max width=\textwidth, max height=0.7\textheight}
 }{
 \end{adjustbox}
}

\title{OOP and Reflection}


\author{Robert Lowe}

\institute[Southeast Missouri State University] % (optional, but mostly needed)
{
  Department of Computer Science\\
  Southeast Missouri State University
}

\date[]{}
\subject{}

\pgfdeclareimage[height=1.0cm]{university-logo}{images/semo-logo}
\logo{\pgfuseimage{university-logo}}



\AtBeginSection[]
{
  \begin{frame}<beamer>{Outline}
    \tableofcontents[currentsection]
  \end{frame}
}


\begin{document}

\begin{frame}
  \titlepage
\end{frame}


% Structuring a talk is a difficult task and the following structure
% may not be suitable. Here are some rules that apply for this
% solution: 

% - Exactly two or three sections (other than the summary).
% - At *most* three subsections per section.
% - Talk about 30s to 2min per frame. So there should be between about
%   15 and 30 frames, all told.

% - A conference audience is likely to know very little of what you
%   are going to talk about. So *simplify*!
% - In a 20min talk, getting the main ideas across is hard
%   enough. Leave out details, even if it means being less precise than
%   you think necessary.
% - If you omit details that are vital to the proof/implementation,
%   just say so once. Everybody will be happy with that.


\begin{frame}{Definitions}
\begin{block}{Reflection}
{\bf Reflection} is when a program can see and/or manipulate its own metadata.
\end{block}
    \begin{description}
        \item<2->[Introspection] -- The act of inspecting metadata.
        \item<3->[Intercession] -- The act of reacting to metadata.
    \end{description}
\end{frame}

\begin{frame}[fragile]{An Example in Java}
\begin{code}
public class Duck
{
    public void sound() {
        System.out.println("Quack!");
    }
}

public class Cow {
    public void sound() {
        System.out.println("Mooo!");
    }
}

public class Car {
    public void sound() {
        System.out.println("Honk Honk!");
    }
}
\end{code}
\end{frame}

\begin{frame}[fragile]{An Example in Java}
\begin{code}
    public static void listen(Object o) {
        try {
            Class cls = o.getClass();
            Method m = cls.getMethod("sound");
            m.invoke(o);
        } catch(Exception e) {
            System.out.println("Could not invoke sound.");
        }
    }
    
    public static void listObject(Object o) {
        Class cls = o.getClass();

        System.out.printf("Class Name: %s\n", cls.getCanonicalName());
        System.out.println("Fields");
        for(Field f : cls.getFields()) {
            System.out.println(f.getName());
        }
        System.out.println("Methods");
        for(Method m : cls.getMethods()) {
            System.out.println(m.getName());
        }
    }
\end{code} 
\end{frame}

\begin{frame}[fragile]{An Example in Java}
\begin{code}
    public static void main(String [] args) {
        Duck duck = new Duck();
        Cow cow   = new Cow();
        Car car   = new Car();

        listen(duck);
        listen(cow);
        listen(car);
        listen(args);

        listObject(duck);
        listObject(args);
    }
\end{code}
\end{frame}

\begin{frame}[fragile]{An Example in Java}
\begin{code}
Quack!
Mooo!
Honk Honk!
Could not invoke sound.
Class Name: Duck
Fields
Methods
sound
wait
wait
wait
equals
toString
hashCode
getClass
notify
notifyAll
Class Name: java.lang.String[]
Fields
Methods
wait
wait
wait
equals
toString
hashCode
getClass
notify
notifyAll
\end{code}
\end{frame}

\begin{frame}[fragile]{An Example in Python}
\begin{code}
class Duck:
    def sound(self):
        print("Quack!")


class Cow:
    def sound(self):
        print("Mooo!")


class Car:
    def sound(self):
        print("Honk honk!")
\end{code}
\end{frame}


\begin{frame}[fragile]{An Example in Python}
\begin{code}
def listen(o):
    sound = getattr(o, 'sound', None)
    if callable(sound):
        sound()
    else:
        print("Could not invoke sound function")

def list_object(o):
    print("Class Name: ", type(o))
    print(dir(o))
\end{code}
\end{frame}


\begin{frame}[fragile]{An Example in Python}
\begin{code}
def main():
    duck = Duck()
    cow = Cow()
    car = Car()

    listen(duck)
    listen(cow)
    listen(car)
    listen([])

    list_object(duck)
    list_object([])

if __name__=='__main__':
    main()
\end{code}
\end{frame}


\begin{frame}[fragile]{An Example in Python}
\begin{code}
Quack!
Mooo!
Honk honk!
Could not invoke sound function
Class Name:  <class '__main__.Duck'>
['__class__', '__delattr__', '__dict__', '__dir__', '__doc__', '__eq__', '__format__', 
 '__ge__', '__getattribute__', '__gt__', '__hash__', '__init__', '__init_subclass__', 
 '__le__', '__lt__', '__module__', '__ne__', '__new__', '__reduce__', '__reduce_ex__', 
 '__repr__', '__setattr__', '__sizeof__', '__str__', '__subclasshook__', '__weakref__', 
 'sound']
Class Name:  <class 'list'>
['__add__', '__class__', '__class_getitem__', '__contains__', '__delattr__', '__delitem__', 
 '__dir__', '__doc__', '__eq__', '__format__', '__ge__', '__getattribute__', '__getitem__', 
 '__gt__', '__hash__', '__iadd__', '__imul__', '__init__', '__init_subclass__', '__iter__', 
 '__le__', '__len__', '__lt__', '__mul__', '__ne__', '__new__', '__reduce__', '__reduce_ex__', 
 '__repr__', '__reversed__', '__rmul__', '__setattr__', '__setitem__', '__sizeof__', '__str__', 
 '__subclasshook__', 'append', 'clear', 'copy', 'count', 'extend', 'index', 'insert', 'pop', 
 'remove', 'reverse', 'sort']

\end{code}
\end{frame}

\begin{frame}{Pros and Cons of Reflection}
    \begin{itemize}
        \item<2-> Reflection violates polymorphism!
        \item<3-> Reflection evades compile-time type checking.
        \item<4-> The primary legitimate use of reflection is in testing and code evaluation.
    \end{itemize}
\end{frame}

\begin{frame}[fragile]{A More Correct Java Program}
\begin{code}
public interface Soundable {
    public void sound();
}
\end{code}
\end{frame}


\begin{frame}[fragile]{A More Correct Java Program}
\begin{code}
public class Duck implements Soundable
{
    public void sound() {
        System.out.println("Quack!");
    }
}

public class Cow implements Soundable {
    public void sound() {
        System.out.println("Mooo!");
    }
}

public class Car implements Soundable  {
    public void sound() {
        System.out.println("Honk Honk!");
    }
}
\end{code}
\end{frame}

\begin{frame}[fragile]{A More Correct Java Program}
\begin{code}
    public static void listen(Soundable o) {
        o.sound();
    }

    public static void main(String [] args) {
        Duck duck = new Duck();
        Cow cow   = new Cow();
        Car car   = new Car();

        listen(duck);
        listen(cow);
        listen(car);
        // Can't do this one thanks to compile-time checking ---> listen(args);

    }

\end{code}
\end{frame}

\begin{frame}{Reflection in C++}
\begin{itemize}
    \item<2-> Compile-Time Introspection is Possible through templates.
    \item<3-> Run-Time Introspection is Impossible.
    \item<4-> There are a variety of macros and techniques which can be used to invade object methods and memory in C++.
    \item<5-> Such methods are implementation dependent, and a very bad idea. 
    \item<6-> So you'll have to Google them yourself! (But please don't use them.)
\end{itemize}
\end{frame}

\end{document}
