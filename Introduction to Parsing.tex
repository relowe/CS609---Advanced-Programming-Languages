% Uncomment for handout
\def\HANDOUT{}


\ifdefined\HANDOUT
\documentclass[handout]{beamer}
\usepackage{pgfpages}
\pgfpagesuselayout{4 on 1}[letterpaper,landscape,border shrink=5mm]
\else
\documentclass{beamer}
\fi

\mode<presentation>
{
  \usetheme{Warsaw}
  \definecolor{sered}{rgb}{0.78, 0.06, 0.18}
  \definecolor{richblack}{rgb}{0.0, 0.0, 0.0}
  \setbeamercolor{structure}{fg=sered,bg=richblack}
  %\setbeamercovered{transparent}
}


\usepackage[english]{babel}
\usepackage[latin1]{inputenc}
\usepackage{times}
\usepackage[T1]{fontenc}
\usepackage{tikz}
\usepackage{graphicx}
\usepackage[export]{adjustbox}
\usepackage{fancyvrb}
\usepackage{amsmath}
\usepackage{amssymb}
\usepackage{esvect}

\makeatletter
\newcommand{\imagesource}[1]{{\centering\hfill\break\hbox{\scriptsize Image Source:\thinspace{\tiny\itshape #1}}\par}}
\newcommand{\image}[3][\@nil]{%
        \def\tmp{#1}%
        \begin{center}
        \ifx\tmp\@nnil
            \includegraphics[max height = 0.55\textheight, max width = \textwidth]{images/#2}
        \else
            \includegraphics[max height = 0.50\textheight, max width = \textwidth]{images/#2}
            \linebreak
            #1
        \fi
        \linebreak
        {\tiny Image Source:\thinspace{\tiny #3}}
        \end{center}
}

\newenvironment{code}{%
 \VerbatimEnvironment
 \begin{adjustbox}{max width=\textwidth, max height=0.7\textheight}
 \begin{BVerbatim}
  }{
  \end{BVerbatim}
 \end{adjustbox}
}


\newenvironment{scaled}{%
 \begin{adjustbox}{max width=\textwidth, max height=0.7\textheight}
 }{
 \end{adjustbox}
}

\title{Introduction to Parsing}


\author{Robert Lowe}

\institute[Southeast Missouri State University] % (optional, but mostly needed)
{
  Department of Computer Science\\
  Southeast Missouri State University
}

\date[]{}
\subject{}

\pgfdeclareimage[height=1.0cm]{university-logo}{images/semo-logo}
\logo{\pgfuseimage{university-logo}}



\AtBeginSection[]
{
  \begin{frame}<beamer>{Outline}
    \tableofcontents[currentsection]
  \end{frame}
}


\begin{document}

\begin{frame}
  \titlepage
\end{frame}

\begin{frame}{Outline}
  \tableofcontents
\end{frame}


% Structuring a talk is a difficult task and the following structure
% may not be suitable. Here are some rules that apply for this
% solution: 

% - Exactly two or three sections (other than the summary).
% - At *most* three subsections per section.
% - Talk about 30s to 2min per frame. So there should be between about
%   15 and 30 frames, all told.

% - A conference audience is likely to know very little of what you
%   are going to talk about. So *simplify*!
% - In a 20min talk, getting the main ideas across is hard
%   enough. Leave out details, even if it means being less precise than
%   you think necessary.
% - If you omit details that are vital to the proof/implementation,
%   just say so once. Everybody will be happy with that.
\section{The Parsing Problem}
\begin{frame}{What is Parsing?}
    \begin{itemize}
        \item Parsing is working the generating problem backwards.
        \item Given a sentence in a language, produce a parse tree for that sentence.
        \item If no parse tree can be found, we reject the 
    \end{itemize}
\end{frame}

\begin{frame}{Basic Approaches to Parsing}
    \begin{description}
        \item[Top-Down] We start with the root of the tree and build toward the leaves.
        \begin{itemize}
            \item Easiest to Implement
            \item Generally Requires Grammar Constraints
        \end{itemize}
        \item[Bottom-Up] We start at the leaves and build toward the root.
        \begin{itemize}
            \item Difficult to Implement, Usually Machine Generated
            \item Has Fewer Grammar Constraints
        \end{itemize}
    \end{description}
\end{frame}

\begin{frame}{A Top-Down Example: \texttt{2+3}}
    \begin{columns}
    \column{0.5\textwidth}
    \begin{enumerate}[<+->]
        \item expr
        \item expr \texttt{+} term
        \item term
        \item num
        \item \texttt{2}
        \item num
        \item \texttt{3}
    \end{enumerate}
    \column{0.5\textwidth}
    \begin{center}
        \begin{scaled}
            \begin{tikzpicture}
                    \node{exp}
                        child {node {expr}
                            child {node {term}
                                child {node {num}
                                    child {node {\texttt{2}}}}
                            }
                        }
                        child {node {\texttt{+}}}
                        child {node {term}
                            child {node {num}
                                child {node {\texttt{3}}}
                            }
                        }
            \end{tikzpicture}
        \end{scaled}
    \end{center}
    \end{columns}
\end{frame}

\begin{frame}{A Bottom-Up Example: \texttt{2+3}} 
    \begin{columns}
    \column{0.5\textwidth}
    \begin{enumerate}[<+->]
        \item \texttt{3}
        \item num
        \item term 
        \item \textt{2}
        \item num
        \item term
        \item expr
        \item expr \texttt{+} term
        \item exp
    \end{enumerate}
    \column{0.5\textwidth}
    \begin{center}
        \begin{scaled}
            \begin{tikzpicture}
                    \node{exp}
                        child {node {expr}
                            child {node {term}
                                child {node {num}
                                    child {node {\texttt{2}}}}
                            }
                        }
                        child {node {\texttt{+}}}
                        child {node {term}
                            child {node {num}
                                child {node {\texttt{3}}}
                            }
                        }
            \end{tikzpicture}
        \end{scaled}
    \end{center}
    \end{columns}
\end{frame}


\section{Parsing Algorithms}
\begin{frame}{The Complexity of Parsing}
    \begin{itemize}
        \item General parsing has complexity $O(n^3)$.
        \item General parsing also has a $O(n^3)$ backtracking requirement.
        \item With a few constraints, we can parse in $O(n)$ time!
        \item Most computer languages can be parsed in linear time.
    \end{itemize}
\end{frame}

\begin{frame}{Classifications of Parsers}  
    Parsers are classified according to the {\bf direction of scanning}, and the {\bf direction of derivation}.
    \begin{description}
        \item[LL] - Left-to-Right Scanning, Left most derivation.
        \item[LR] - Left-to-Right Scanning, Right most derivation.
    \end{description}
    Parsers are further classified by how many tokens they must {\bf look ahead}.
    \begin{description}
        \item[LL$(k)$] - LL Parser with constant $k$ tokens of look-ahead.
        \item[LR$(k)$] - LR Parser with constant $k$ tokens of look-ahead.
        \item[LALR] - A simplified LR$(k)$ parser.
    \end{description}
\end{frame}
\begin{frame}[fragile]{Recursive Descent Parsing}
    \begin{itemize}
        \item Recursive Descent Parsers are derived from the BNF.
        \item Each non-terminal is expressed as a mutually recursive function.
        \item RD Parsers have no backtracking.
        \item Only apply to $LL(1)$ grammars!
    \end{itemize}
\end{frame}

\begin{frame}{LALR Parsers}
    \begin{itemize}
        \item Most typical example is the LALR$(1)$ parser.
        \item Use a state table to maintain a list of possible derivations.
        \item Performs a {\bf shift-reduce} operation in order to decrease possiblities list.
        \item These are generally machine generated.
    \end{itemize}
\end{frame}

\begin{frame}{Compiler Compilers --- Parser Generators}
    \begin{itemize}
        \item Compiler Compilers generate parsers from a (typically) BNF-like syntax.
        \item Most commercial compilers are created using these tools.
        \item Notable Examples:
        \begin{itemize}
            \item YACC - Yet Another Compiler Compiler (C Parser Generator)
            \item JACC - Just Another Compiler Compiler (Java Parser Generator)
            \item PLY - Python-Lex-Yacc (Python Lexer and Parser Generator)
        \end{itemize}
    \end{itemize}
\end{frame}

\begin{frame}[fragile]{YACC Example}
\begin{code}
%left '+' '-'
%left '*' '/' 

%% 
expr:    '(' expr ')'    { $$ = $2; }
         | expr '*' expr { $$.a = $1.a * $3.a; }
         | expr '/' expr { $$.a = $1.a / $3.a; }
         | expr '+' expr { $$.a = $1.a + $3.a; }
         | expr '-' expr { $$.a = $1.a - $3.a; }
\end{code}
\end{frame}


\begin{frame}{Reading Assignment}
    Read the rest of Chapter 4 (4.3 to the end of the chapter.)
\end{frame}

\end{document}

