% Uncomment for handout
\def\HANDOUT{}


\ifdefined\HANDOUT
\documentclass[handout]{beamer}
\usepackage{pgfpages}
\pgfpagesuselayout{4 on 1}[letterpaper,landscape,border shrink=5mm]
\else
\documentclass{beamer}
\fi

\mode<presentation>
{
  \usetheme{Warsaw}
  \definecolor{sered}{rgb}{0.78, 0.06, 0.18}
  \definecolor{richblack}{rgb}{0.0, 0.0, 0.0}
  \setbeamercolor{structure}{fg=sered,bg=richblack}
  %\setbeamercovered{transparent}
}


\usepackage[english]{babel}
\usepackage[latin1]{inputenc}
\usepackage{times}
\usepackage[T1]{fontenc}
\usepackage{tikz}
\usepackage{graphicx}
\usepackage[export]{adjustbox}
\usepackage{fancyvrb}
\usepackage{amsmath}
\usepackage{amssymb}
\usepackage{esvect}

\makeatletter
\newcommand{\imagesource}[1]{{\centering\hfill\break\hbox{\scriptsize Image Source:\thinspace{\tiny\itshape #1}}\par}}
\newcommand{\image}[3][\@nil]{%
        \def\tmp{#1}%
        \begin{center}
        \ifx\tmp\@nnil
            \includegraphics[max height = 0.55\textheight, max width = \textwidth]{images/#2}
        \else
            \includegraphics[max height = 0.50\textheight, max width = \textwidth]{images/#2}
            \linebreak
            #1
        \fi
        \linebreak
        {\tiny Image Source:\thinspace{\tiny #3}}
        \end{center}
}

\newenvironment{code}{%
 \VerbatimEnvironment
 \begin{adjustbox}{max width=\textwidth, max height=0.7\textheight}
 \begin{BVerbatim}
  }{
  \end{BVerbatim}
 \end{adjustbox}
}


\newenvironment{scaled}{%
 \begin{adjustbox}{max width=\textwidth, max height=0.7\textheight}
 }{
 \end{adjustbox}
}

\title{Traversing the Parse Tree}


\author{Robert Lowe}

\institute[Southeast Missouri State University] % (optional, but mostly needed)
{
  Department of Computer Science\\
  Southeast Missouri State University
}

\date[]{}
\subject{}

\pgfdeclareimage[height=1.0cm]{university-logo}{images/semo-logo}
\logo{\pgfuseimage{university-logo}}



\AtBeginSection[]
{
  \begin{frame}<beamer>{Outline}
    \tableofcontents[currentsection]
  \end{frame}
}


\begin{document}

\begin{frame}
  \titlepage
\end{frame}

\begin{frame}{Outline}
  \tableofcontents
\end{frame}


% Structuring a talk is a difficult task and the following structure
% may not be suitable. Here are some rules that apply for this
% solution: 

% - Exactly two or three sections (other than the summary).
% - At *most* three subsections per section.
% - Talk about 30s to 2min per frame. So there should be between about
%   15 and 30 frames, all told.

% - A conference audience is likely to know very little of what you
%   are going to talk about. So *simplify*!
% - In a 20min talk, getting the main ideas across is hard
%   enough. Leave out details, even if it means being less precise than
%   you think necessary.
% - If you omit details that are vital to the proof/implementation,
%   just say so once. Everybody will be happy with that.
\section{Introduction to Parse Tree Evaluation}
\begin{frame}[fragile]{An Example Parse Tree}
\begin{columns}
\column{0.5\textwidth}
\begin{code}
BEGIN
    NUMBER x
    
    x := 2 * (3+2)
    PRINT X
END
\end{code}

\column{0.5\textwidth}
\begin{scaled}
\begin{tikzpicture}
    \node {Block}
        child { 
            node { Decl }
        }
        child { node {\texttt{:=}}
            child { node {\texttt{x}}}
            child { node {\texttt{*}}
                child {node {\texttt{2}}}
                child {node {\texttt{+}}
                    child {node {\texttt{3}}}
                    child {node {\texttt{2}}}
                }
            }
        }
        child { node {\texttt{PRINT}}
            child{node {\texttt{x}}}
        }
\end{tikzpicture}
\end{scaled}
\end{columns}
\end{frame}

\begin{frame}{Traversal and Evaluation}
\begin{columns}
\column{0.5\textwidth}
\begin{itemize}
    \item Traverse the tree in a pre-order traversal.
    \item Visit the children to get the operands.
    \item Then we evaluate our operation and return it higher up the tree.
\end{itemize}
\column{0.5\textwidth}
\begin{scaled}
\begin{tikzpicture}
    \node {Block}
        child { 
            node {(1) Decl }
        }
        child { node {(8) \texttt{:=}}
            child { node {(2) \texttt{x}}}
            child { node {(7)\texttt{*}}
                child {node {(3) \texttt{2}}}
                child {node {(6)\texttt{+}}
                    child {node {(4) \texttt{3}}}
                    child {node {(5) \texttt{2}}}
                }
            }
        }
        child { node {(10)\texttt{PRINT}}
            child{node {(9) \texttt{x}}}
        }
\end{tikzpicture}
\end{scaled}
\end{columns}
\end{frame}

\begin{frame}{Building the Parse Tree Structure}
    \begin{itemize}
        \item Each parser rule function returns a node.
        \item We build a list of child nodes and then return our node as we make recursive calls.
        \item These nodes should all be something which can be evaluated.
        \item The start symbol returns the parse tree of the program.
    \end{itemize}
\end{frame}

\begin{frame}[fragile]{Psuedocode for Tree Evaluation}
\begin{code}
def evaluate(tree):
    operands = empty list
    for each child of tree:
        operands.append(evaluate(child))
    perform the operation specified by tree on the operands
    return the result
\end{code}
\end{frame}

\section{The lis.py Interpreter}
\begin{frame}[fragile]{The \texttt{lis.py} Syntax}
\begin{code}
< s_expression >    ::= < atomic_symbol >
                        | < list >


< list >            ::= "(" < s_expression > < list_part > ")"


< list_part>        ::= < s_expression > < list_part >
                        | ""


< atomic_symbol >   ::= < letter > < atom_part >


< atom_part >       ::= < symbol > < atom_part >
                        | ""

< symbol > is any symbol other than "(" or ")"
\end{code}
\end{frame}

\begin{frame}[fragile]{Example \texttt{lis.py} code}
    \begin{block}{Basic Syntax}
    \begin{code}
    (operation arg1 arg2 ... argn)
    \end{code}
    \end{block}
    
    \begin{code}
   
    
(display (quote (hello world)))
(display (+ 2 2))
(display (* 2 (+ 3 2)))
    \end{code}
\end{frame}

\begin{frame}[fragile]{Lexer}
\begin{code}
def tokenize(s: str) -> list[str]:
    "Convert a string into a list of tokens."
    return s.replace('(', ' ( ').replace(')', ' ) ').split()
\end{code}
\end{frame}

\begin{frame}[fragile]{Parser}
\begin{code}
def parse(program: str) -> Expression:
    "Read a Scheme expression from a string."
    return read_from_tokens(tokenize(program))

def read_from_tokens(tokens: list[str]) -> Expression:
    "Read an expression from a sequence of tokens."
    if len(tokens) == 0:
        raise SyntaxError('unexpected EOF while reading')
    token = tokens.pop(0)
    if '(' == token:
        exp = []
        while tokens[0] != ')':
            exp.append(read_from_tokens(tokens))
        tokens.pop(0)  # discard ')'
        return exp
    elif ')' == token:
        raise SyntaxError('unexpected )')
    else:
        return parse_atom(token)
        
def parse_atom(token: str) -> Atom:
    "Numbers become numbers; every other token is a symbol."
    try:
        return int(token)
    except ValueError:
        try:
            return float(token)
        except ValueError:
            return Symbol(token)
\end{code}
\end{frame}

\begin{frame}[fragile]{Environments and Pre-Defined Functions}
\begin{code}
class Environment(ChainMap[Symbol, Any]):
    "A ChainMap that allows changing an item in-place."

    def change(self, key: Symbol, value: object) -> None:
        "Find where key is defined and change the value there."
        for map in self.maps:
            if key in map:
                map[key] = value  # type: ignore[index]
                return
        raise KeyError(key)


def standard_env() -> Environment:
    "An environment with some Scheme standard procedures."
    env = Environment()
    env.update(vars(math))   # sin, cos, sqrt, pi, ...
    env.update({
            '+': op.add,
            '-': op.sub,
            '*': op.mul,
            
            ...
            
    See lis.py for the rest
\end{code}
\end{frame}

\begin{frame}[fragile]{Evaluation}
\begin{code}
def evaluate(exp: Expression, env: Environment) -> Any:
    "Evaluate an expression in an environment."
    if isinstance(exp, Symbol):      # variable reference
        return env[exp]
# end::EVAL_IF_TOP[]
    elif not isinstance(exp, list):  # constant literal
        return exp
# tag::EVAL_IF_MIDDLE[]
    elif exp[0] == 'quote':          # (quote exp)
        (_, x) = exp
        return x
    elif exp[0] == 'if':             # (if test conseq alt)
        (_, test, consequence, alternative) = exp
        if evaluate(test, env):
            return evaluate(consequence, env)
        else:
            return evaluate(alternative, env)
    elif exp[0] == 'lambda':         # lambda
        (_, parms, *body) = exp
        return Procedure(parms, body, env)
    elif exp[0] == 'define':
        (_, name, value_exp) = exp
        env[name] = evaluate(value_exp, env)
# end::EVAL_IF_MIDDLE[]
    elif exp[0] == 'set!':
        (_, name, value_exp) = exp
        env.change(name, evaluate(value_exp, env))
    else:                          # (proc args)
        (func_exp, *args) = exp
        proc = evaluate(func_exp, env)
        args = [evaluate(arg, env) for arg in args]
        return proc(*args)
\end{code}
\end{frame}

\begin{frame}{Reading Assignment}
(How to Write a (Lisp) Interpreter (in Python)) by Peter Norvig\newline
\url{http://www.norvig.com/lispy.html}
\end{frame}
\end{document}

